\section*{Recap}
\subsection*{Remarks on $GI$}
\begin{itemize}
    \item For any graph $G$, the set of all $(G,G')$, where $G'$ is isomorphic to $G$ is an equivalence relation.
    \item For any graph $G$: $(\pi(G))_{\pi\leftarrow\mathbb{U}}\approx\mathbb{Iso(G)}$ - meaning uniformly sampling a permutation (over the set of nodes) of $G$ and applying it to $G$ will yield a uniformly sampled graph from the same equivalence partition.
\end{itemize}
\subsection*{$GNI\subseteq IP$}
To prove this we will present an interactive protocol that runs on input $G_0, G_1$:
\begin{itemize}
    \item $V$ samples $b\leftarrow \{0,1\},\pi\leftarrow \{\texttt{permutations}\}$.
    \item $V$ sends $G=\pi(G_b)$.
    \item $P$ checks: if $G\approx G_0$, $b'=0$ else $b'=1$.
    \item $P$ sends $b'$ to $V$.
    \item $V$ accepts iff $b=b'$.
\end{itemize}
Now we prove that this interactive protocol is indeed an interactive proof for $GNI$:
\begin{enumerate}
    \item Soundness:
    Let $G_0\approx G_1$ and let $P^*$ be a (possibly) cheating prover.\\
    If $b=0$ then $G$ is uniformly distributed over all graphs isomorphic to $G_0$.\\ If $b=1$ then $G$ is uniformly distributed over all graphs isomorphic to $G_1$, but due to transitivity, this is the same as the prior case, meaning in any case, $G$ is distributed uniformly over the equivalence to $G_0$.\\
    Since this is the case, no information has been passed from $V$ to $P^*$ by sending $G$ ($P^*$ could have just sampled a graph from $Iso(G_0)$ itself!), $P^*$ could not do more than guess $b$, and it's probability for success is no more than $\frac{1}{2}$.
    \item Completeness: If the graphs are not isomorphic, $G\approx G_0$ iff $b=0$, which will mean $b=b'$ is always the case and $V$ always accepts.
    \item Runtime: $V$ does nothing of high complexity, $P$ is unbounded.
\end{enumerate}


\section*{Equivalence between $IP$ and $AM[Poly]$}
\subsection*{Lemma: Set Lower Bound Protocol}%TODO: add protocol and proof

\subsection*{Theorem}
\begin{itemize}
    \item $IP[k]$: $k$-round interactive proof.
    \item $AM[k]$: $k$-round public coin interactive proof.
\end{itemize}
Claim (no proof):
$AM[k]\subseteq IP[k]$.\\
Theorem:
$IP[k]\subseteq AM[k+2]$.\\
Corollary: $IP=AM[Poly]$.\\
Theorem Proof:\\
Let $L$ be a language with an $IP$ which satisfies the following assumptions:
\begin{enumerate}
    \item $k=2$.
    \item Perfect Completeness.
    \item There is a set of $N$ of possible verifier messages which is constant for any specific protocol input $x$.
    \item The message $m$ sent by $V$ is uniformly distributed over the $N$ possible messages that could be sent.
    \item The soundness error is $\frac{1}{100}$.
\end{enumerate}
We want to show that $L$ is in $AM[k+2]$.\\
First, note how our $IP$ can be described as follows when running on input $x$:\\
\begin{itemize}
    \item $V$ samples some $r\leftarrow\{0,1\}$.
    \item $V$ calculates some $m=V_1(x,r)$ for some efficient function $V_1$.
    \item $V$ sends $m$ to $P$.
    \item $P$ sends back some $\pi$.
    \item $V$ accepts iff $V_2(x,\pi,r)=1$ for some efficient function $V_2$.
\end{itemize}
Define:
\[R_m=\{r'\mid m=V_1(x,r')\}, S_m=\{r'\in R_m\mid V_2(x,\pi,r')=1\}\]
Now consider an alternative protocol, which is the same as the one above, only $V$ also samples $r'\leftarrow R_m$ and accepts iff $V_2(x,\pi, r')$.

Note how $|R_m|=\frac{2^l}{N}$, and if $x\in L$ then $S_m=R_m$.
Claim (missing proof):\\%TODO add proof
if $x\in L$ then $\mathbb{E}(|S_m|)=\frac{2^l}{N}$ and otherwise $\mathbb{E}(|S_m|)\leq \frac{1}{100}\frac{2^l}{N}$.\\

\subsection*{Pairwise Independent Hash Functions}
Definition:
A set of hash functions $\{h:X\leftarrow Y\}$ is pairwise independent if for all $x_1, x_2\in X$ where $x_1\neq x_2$ and $y_1,y_2\in Y$, they satisfy:
\[Pr_{h\leftarrow H}[h(x_1)=y_1\wedge h(x_2)=y_2]=\frac{1}{|Y|^2}\]
Construction:\\
Let $\mathbb{F}$ be a finite field.
Given $a,b\in\mathbb{F}: h_{a,b}(x)=ax+b$.\\
Let $H=\{h_{a,b}\mid a,b\in\mathbb{F}\}$.\\
Construction proof:\\
Let $x_1,x_2,y_1,y_2$ as described in definition.\\
\[
    Pr_{a,b}[ax_1+b=y_1\wedge ax_2+b=y_2]
    =Pr_{a,b}[
\begin{pmatrix}
x_1 & 1\\
x_2 & 1
\end{pmatrix}
\begin{pmatrix}
a\\
b
\end{pmatrix}
=
\begin{pmatrix}
y_1\\
y_2
\end{pmatrix}
]\]
\[
 =Pr_{a,b}[
\begin{pmatrix}
a\\
b
\end{pmatrix}
=
\begin{pmatrix}
x_1 & 1\\
x_2 & 1
\end{pmatrix}^{-1}
\begin{pmatrix}
y_1\\
y_2
\end{pmatrix}
]=\frac{1}{|\mathbb{F}|^2}\]
For the last transition: Note how the right value of the equality (in the probability) is just a constant value, so we just need to randomly sample $a,b$ to be some spesific values within a finite field.
