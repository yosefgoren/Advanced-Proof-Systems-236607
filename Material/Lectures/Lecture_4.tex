\section*{$coNP\subseteq IP$}
\subsection*{Arithmetization}
We interested in a reduction from a $coNP$ problem to an arithmetic
problem.\\

\underline{The Sumcheck Problem:}\\
Parameters: a finite field $\mathbb{F}$, and $n,d\in\mathbb{N}$.\\
Input: $Q:\mathbb{F}^n\leftarrow\mathbb{F}$, $\alpha\in\mathbb{F}$.
Problem: does $\sum_{x\in\{0,1\}^n}Q(x)=\alpha$ ?\\

\underline{The reduction}
We will be reducing from $coNP$ to the sumcheck problem
by reducing $3-CNF$ to it (since $3-CNF$ is $coNP$-complete).\\
Let $\phi\in 3-CNF$ with $n$ variables and $m$ clauses.\\
We will start the construction by translating each building block
of $3-CNF$ formulas and expressing it in terms of polynomials:
\begin{enumerate}
	\item $\phi(x_1,x_2,...,x_n)=x_1\longrightarrow p(x_1,x_2,...,x_n)=x_1$
	\item $x_1\wedge x_2 \longrightarrow x_1\cdot x_2$
	\item $x_1\wedge \neg x_3\longrightarrow x_1\cdot(1-x_3)$
	\item $(x_1)\vee (x_2)=\neg((\neg x_1)\wedge(\neg x_2))\longrightarrow 1-(1-x_1)\cdot(1-x_2)$
\end{enumerate}
Lemma: for every $3-CNF$ formula $\phi$ on $m$ clauses and a finite field $\mathbb{F}$,
there exists a ploynomial $p:\mathbb{F}^n\rightarrow\mathbb{F}$ with degree $O(m)$ s.t.
$\phi=p$.\\
Furthermore, given $\phi$ and $z\in\mathbb{F}^n$, $p(z)$ can be
evaluated in $poly(n,m,log(|\mathbb{F}|))$ time.\\

Given the lemma, we can esaly solve $3-CNF$, by constructing the
appropriate polynomial $p$, and evaluating the sum of the polynomial:
and checking if the sum is $0$.

\subsection*{$IP$ for the sumcheck problem}
We will show that the sumcheck problem is in $IP$ by the following protocol
on the inputs $Q:\mathbb{F}^n\rightarrow\mathbb{F}$, $\alpha\in\mathbb{F}$:
\begin{itemize}
	\item $P$ defines $Q_1:\mathbb{F}\rightarrow \mathbb{F}$ as
	\[
		Q_1(\lambda)=\sum_{x_2,x_3,...,x_n\in\{0,1\}}Q(\lambda,x_2,x_3,...,x_n)
	\]
	And sends it as $\hat{Q}_0$.
	\item $V$ rejects if $\hat{Q}_0(0)+\hat{Q}_0(1)\neq \alpha$.
	\item for $i\in[n]$:
	\begin{itemize}
		\item  $V$ samples $r_i\leftarrow \mathbb{F}$ and sends it.
		\item $P$ defines:
		\[Q_i(\lambda)
		=\sum_{x_{i+2},x_{i+3},...x_n\in\{0,1\}}(r_1,...,r_i,\lambda ,x_{i+1},...x_n)\]
		And sends it as $\hat{Q}_i$.
		\item $V$ rejects if $\hat{Q}_{i}(0)+Q(1)\neq\hat{Q}_{i-1}(r_i)$
	\end{itemize}
	\item $V$ accepts.
\end{itemize}

\underline{Completness:}
Here we know that: $\sum_xQ(x)=\alpha$. So we can write:
\[
	\hat{Q}_1(0)+\hat{Q}_1(1)=\hat{Q}_1(0)+\hat{Q}_1(1)
	=\sum_{x_2,x_3,...,x_n}Q(0,x_2,x_3,...,x_n)+\sum_{x_2,x_3,...,x_n}Q(1,x_2,x_3,...,x_n)
\]\[
	=\sum_xQ(x)=\alpha
\]
Thus first check passes. Additionally:
\[
	\hat{Q}_1(r_1)+\hat{Q}_1(r_1)
	=\sum_{x_2,x_3,...,x_n}Q(r_1,x_2,x_3,...,x_n)
\]
By induction appect w.p. 1.\\

\underline{Soundness:}
Assume $\sum_x\neq\alpha$ and let $P^*$ be a cheating prover.\\
WLOG $P^*$ always sends $\hat{Q}_1\neq Q_1$.\\
Observe that $Q_1$ and $\hat{Q}_1$ are polynomials of degree $d$
so with all but a probability of $\frac{d}{|\mathbb{F}|}$ - we we have $Q_1(r_1)\neq \hat{Q}_1(r_1)$.\\
Assume $Q_1(r_1)\neq \hat{Q}_1(r_1)$, thus:
\[
	\hat{Q}_1(r_1)\neq\sum_{x_2,x_3,...,x_n}Q(r_1,x_2,x_3,...,x_n)	
\]

\[
	Pr[V \texttt{ accepts}]\leq 
	Pr[Q_1(r_1)=\hat{Q}_1(r_1)]
	+Pr[V_{n-1} \texttt{ accepts}\mid Q_1(r_1)=\hat{Q}_1(r_1)]
\]\[
	\leq \frac{d}{|\mathbb{F}|}
	+\frac{d\cdot(n-1)}{|\mathbb{F}|}
	=\frac{d\cdot n}{|\mathbb{F}|}
\]
