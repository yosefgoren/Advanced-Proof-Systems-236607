\section*{Reminder}
\underline{Lemma (sumcheck):}\\
There exists an IP between $P$ and $V$ where $P$
gets as input a polynomial $Q:\mathbb{F}^n\rightarrow\mathbb{F}$,
of individual degree $d$ and a value $\alpha\in\mathbb{F}$
and $V$ gets 'oracle access' to $Q$ and $\alpha$ explicitly.

\begin{itemize}
	\item Completeness:
	\[\left[
		\sum_{x\in\{0,1\}^n}Q(x)=\alpha
	\right]
	\Rightarrow V \texttt{ accepts w.p. }1\]
	\item Soundness:
	\[\left[
		\sum_{x\in\{0,1\}^n}Q(x)\neq\alpha
	\right]
	\Rightarrow\left[
		\forall P^*, Pr[V \texttt{ accepts with }P^*]\leq\frac{nd}{|\mathbb{F}|}
	\right]\]
\end{itemize}

\underline{Lemma:}
For every $3CNF; \phi:\onz^n\rightarrow\onz^n$
on $m$ clauses and field $\fld$ %TODO: complete the lemma from picture taken of board.

\section*{Probobalistically Checkable Proofs - $PCP$}
\subsection*{Definition}
We say that a $PPT$ machine $V$ is a $PCP$ verifier
for $L$ if:
\begin{enumerate}
	\item Completeness:
	\[
		\left[x\in L\right]
		\Rightarrow \exists\pi: Pr[V^\pi(x)=1]=1
	\]
	\item Soundness:
	\[
		\left[x\notin L\right]
		\Rightarrow \forall\pi^*: Pr[V^{\pi^*}(x)=1]\leq\frac{1}{2}
	\]
	\item Parameters:
	\begin{itemize}
		\item Query complexity - number of queries to $\pi$, denoted with $q$.
		\item Proof length - $|\pi|$, denoted with $l$.
		\item Randomness length - number of random bits sampled by the verifier, denoted with $r$.
	\end{itemize}
\end{enumerate}
We denote $PCP(q,r)$ the class of languages
with $PCP$ with query complexity $q$ and randomness length $r$.

\subsection*{Initial Properties \& Conclusions}
\begin{enumerate}
	\item
	Claim: $l\leq q\cdot 2^r$
	\item 
	Claim: $NP\subseteq PCP(poly, 0)$, more formally:
	\[
		NP\subseteq \bigcup_{p\in poly}PCP(p,0)	
	\]
	\item
	Claim: if $L$ has an interactive proof in which each prover/verifier message has length $a/b$ respectively.
	and with $k$ rounds, then $L$ has a $PCP$
	with length $a\cdot 2^{kb}$ and query complexity $a\cdot k$.
	\item
	Claim: $PCP(q,r)$ has $NP$ proof of length $q\cdot 2^r$.
\end{enumerate}
3. results with that any $L\in PSPACE$ has a proof
of exponential length that can be checked by a polynomial verifier, with a polynomial
number of queries.

\section*{PCP theorem: $NP = PCP(O(1), O(log(n)))$}
We will see the proof of this claim in a few steps:
\begin{itemize}
	\item $NP \subseteq PCP(O(1), poly(n))$
	\item $NP\subseteq PCP(O(log(n)^2), O(log(n)))$
	\item Combining the two claims above to get the theorem.
\end{itemize}

\subsection*{Hardness of Approxiamtion (Application)}
$GapSAT_\epsilon$: Accept satisfiable $CNF$ formulas,
and reject formulas s.t. every assignment satisfies at most $\epsilon$ clauses.
(here $\epsilon \in[0,1]$ is the proportion of clauses satisfied).\\

\underline{Theorem:}\\
\[
	\exists\epsilon >0: GapSAT_\epsilon\in NPC
\]
\underline{Proof:}\\
Let $L\in NP$.\\
Given $x\in\onz^n$, if $x\in L$,
there exists a $PCP$ proof $\pi$ for $x$ (The PCP theorem).\\

For a given $x, p$ and $V_{PCP}$, define $V_{x,p}:\onz^q\rightarrow\onz$
as follows:\\
There exists a CNF computing $V_{x,p}$ with $2^q$ clauses.\\
Consider the following CNF:
$
	\bigwedge_{p}V_{x,p}
$
\begin{itemize}
	\item if $x\in L$, it is easy to see that $V_{x,p}$ is satisfied.
	\item 
	if $x\notin L$ for any assignment $\pi$ for half of $p$'s $V_{x,p}(\pi)=0$
	so overall, at-least $\frac{1}{2}\frac{1}{2^q}$ of the clauses 
	are not satisfied.
\end{itemize}
With $\epsilon=1-\frac{1}{2^{q+1}}$, we have that
$L\in GapSAT_\epsilon$.