\section*{Public Coin Protocol For Set Size}
\subsection*{Lemma (Set LB)}
Exists a protocol $(P,V)$ such that:
Given membership access to $s\subseteq U, t\subseteq \mathbb{N}$:
\begin{enumerate}
	\item Completeness:
	\[|s|\geq t\Rightarrow Pr[V \texttt{ accepts}]\geq \frac{2}{3}\]
	\item Soundness:
	\[\forall P^*, |s|\leq \frac{t}{100}\Rightarrow Pr[V \texttt{ accepts}]\leq \frac{1}{3}\]
\end{enumerate}

% TODO: add pairwise independent function definition

\subsection*{The Protocol}
\begin{enumerate}
	\item $V$ samples $h\leftarrow H$ where $H=\{h:U\rightarrow [t]\}$
	\item $P$ finds $x\in S$ s.t. $h(x)=17$ and sends it to $V$ (if none exist, $P$ fails).
	\item accepts iff $x\in S$ and $h(x) = 17$
\end{enumerate}

\subsection*{Proof}
For $x\in U, h\in H$, denote $E_x$ if $h(x)=17$.\\
Soundness: $|S|\leq\frac{t}{100}$.\\
\[
	Pr_{h\leftarrow H}[\exists x\in S, h(x)=17]
	=Pr_{h\leftarrow H}[\bigcup_{x\in S}E_x]
	\leq\sum_{x\in S}Pr[E_x]
	=\frac{|S}{t}
	\leq\frac{1}{100}
\]
Completeness: $|S|=t$.\\
\[
	Pr_{h\leftarrow H}[\exists x\in S, h(x)=17]
	=Pr_{h\leftarrow H}[\bigcup_{x\in S}E_x]
	\geq\sum_{x\in S}Pr[E_x]-Pr_{x,x'\in S, x<x'}[E_x\cap E_{x'}]
\]\[
	=\frac{|S|}{t}-\binom{|S|}{2}\frac{1}{t^2}
	\geq \frac{|s|}{t}-\frac{|s|^2}{2t^2}=\frac{1}{2}
\]

\section*{Zero Knowlage Proofs ($ZKP$)}
\subsection*{$ZKP$ protocol for $GI$}
Given input $G_0, G_1$:
\begin{itemize}
	\item $P$ finds a permutation $\psi\in S_n$ s.t. $G_0=\psi(G_1)$.
	\item $P$ uniformly samples a random permutation $\pi\leftarrow S_n$.
	\item $P$ sends $G=\pi(G_0)$ to $V$.
	\item $V$ samples $b\leftarrow \{0,1\}$ and sends it to $P$.
	\item $P$ defines $\sigma=\pi$ if $b=0$ else $\sigma=\pi\circ\psi$ and sends it to $V$.
	\item $V$ accepts iff $\sigma(G_b)=G$.
\end{itemize}

\subsection*{$HV-ZKP$ definition}
An interactive proof $(P,V)$ is a honest-verifier zero knolage proof of $L\subseteq\{0,1\}^*$ if
there exists a $PPT$ algorithm $S$ called 'the simulator' for which:
$\forall x\in L:$
The following distributions are 'similar':
\begin{itemize}
	\item $View^{x,p}(x):=(\texttt{the input }x,\texttt{ randomized coins }r, \texttt{ communications transcript})$
	\item $S(x) \texttt{ (the output of the simulator)}$
\end{itemize}
Here 'similar' can mean two different things:
\begin{itemize}
	\item If it means identiacal, meaning these two distributions are the same distribution,
	we call the protocol a \textbf{perfect $ZKP$}.
	\item If it means that the distributions are computationally indistinguishable, we call the protocol a \textbf{statistical $ZKP$},
	the definition for 'computationally indistinguishable' is in the next tutorial.
\end{itemize} 


The idea here is that since the verifier $V$ could have seen
everything which is seen during the interactive protocol $(P,V)$ by running the simulator $S$,
the verifier did not actually learn anything from it.

The reason this is 'honset-verifier' is because this definition
does not catch the case where the verifier 'cheats' and does not
run the protocol $V$, and might be able to extract information $P$ by doing it.


\subsection*{$ZKP$ definition}
An interactive proof $(P,V)$ is a zero knowlage proof for $L$
if:
\[\forall x,V^*,\exists S: View^{P,V^*}(x)\approx S(x)\]

Like to 'similar' from the honest verifier definition,
'$\approx$' can mean either 'identical' or 'computational indistinguishable';
with the former being a perfect $ZKP$ and the latter being a statistical $ZKP$. We sometimes denote computational indistinguishability by $X\approx_CY$
and perfect indistinguishability with $X\approx_PY$.\\

In words, the difference from the honest-verifier zero ZKP is
that now the simulator should be able to simulate anything that
ANY verifier protocol manages to extract, meaning that regardless
of the verifier protocol - no information is extracted.

\subsection*{$HZ-ZKP$ proof for $GI$ protocol above}
\[
	View:=(G_0,G_1,G,b,\sigma), G=\sigma(G_b)	
\]
Define the following protocol (simulator); on input $G_0, G_1$ do:
\begin{itemize}
	\item Sample $\sigma \leftarrow S_n$
	\item Sample $b\leftarrow \{0,1\}$
	\item Define $G=\sigma(G_b)$
	\item return $(G_0, G_1, G, b, \sigma)$
\end{itemize}
Now consider each cell in the output tuple of the two distributions ($View$ and $S(G_0,G_1)$),
$G_0,G_1$ are constant and always the same. $b\sigma$ are sampled uniformly from
their ranges in both cases.
So far there is no corralation between the cells.
$G$ is defined exactly given the rest of the variables meaning
that since the rest of the variables are the same for both distributions
$G$ is also the same. Furthermore, since the correlations were identical so far,
the new correlations are too.

These proofs often (such as in this case) follow the pattern of
first showing that given a subset of the cells of the distribution tuples
are the same due to being uniformly sampled respectively ($(G_0,G_1,b,\sigma)$ in this case),
then showing how the rest of the cells ($G$ in this case)
are a deterministic function of the other cells,
and so the addition of these new cells does not impare with the 
distributions being the same.\\
Formally, if we have distributions $D_1, D_2$ and a (deterministic) function $f$:
\[
	D_1\approx D_2
	\Rightarrow
	(D_1, f_{d\leftarrow D_1}(d))\approx (D_2, f_{d\leftarrow D_2}(d))
\]

\subsection*{$NP\subseteq ZKP$ (proof concept)}
In the lecture, proof by picture is shown for $3COL\in NPC$
has a zero knowlage proof. The main course of the proof consists of
the prover commiting to a spesific permutation of the colors on
a graph (without showing the verifier what these are), and allowing
the verifier to choose a sepsific edge and see that indeed
the coloring on both sides of that edge differ.
For each iteration a cheating prover has some chance of failing due
to the verifier haveing a chance to guess an edge where the two sides
have the same color. Furthermore, the protocol is zero knowlage
since the actual colors seen in each iteration are just two random colors.