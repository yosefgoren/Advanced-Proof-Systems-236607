\section{Recap}
\subsection{$P$ - Polynomial (Class)}
$L\in\{0,1\}^*$ is also in $P$ if there exists an efficient algorithm which decides it.

\subsection{$NP$ - Nondeterministic Polynomial}
$L\in\{0,1\}^*$ is also in $NP$ if there exists an efficient algorithm $V$ and a polynomial $p$ which follow:

\begin{enumerate}
    \item Completeness: $\forall x\in L, \exists y: V(x,y) = 1 \wedge |y|<p(|x|)$
    \item Soundness: $\forall x\notin L \forall y: V(x,y) \neq 1 \vee |y| \geq p(|x|)$
\end{enumerate}

\subsection{$PPT$ - Probobalistic Polynomial Time}
This is a class of algorithms which must run in time polynomial to the size of their input, but also - must be capable of randomization, or 'flipping coins'.

\subsection{$IP$ - Interactive Proof}
A key difference between an Interactive Proof and a proof for an NP proof is that the latter necessarily requires the prover to provide the verifier with something he can use to prove the truth of the calim to others.\\

We denote $(P,V)(x)$ to be the output of $V$ (verifier) after the interaction between $P$ and $V$ on the input $x$. Both $P$ and $V$ can be thought of as $PPT$ algorithms or programs which are capable of communicating with one another.\\

These interactions are often described with an interaction diagram:
\begin{itemize}
    \item $P$ sends to $V$ something
    \item $V$ sends to $P$ something else
    \item \dots
    \item $V$ accepts iff \dots
\end{itemize}

Formal Definition:
We say that $L\in IP$ if there exists a polynomial algorithm $V$, an unbounded algorithm $P$ and some constant $c\in (0.5, 1]$ s.t.
\begin{enumerate}
    \item Completeness: $\forall x\in L, Pr[(P,V)(x)=1]>c$
    \item Soundness: if $x\notin L,\forall P^*\in\mathbf{M}, Pr[(P^*,V)(x)=1]<1-c$
\end{enumerate}
Note: $\mathbf{M}$ denotes the set of turing machines.

%todo: add definition for Perfect Completness

\section{Equivalence of $IP$ separation constants}
\subsection{Iterative Runs}
Given an $IP$ protocol $(P,V)$, let $(P^k,V^k)$ be the protocol obtained by running $(P,V)$ k times sequentially. $V^k$ accepts iff in all iterations $V$ accepted.

\subsection{Lemma}
if $(P,V)$ is $IP$ with perfect completeness then for every polynomial $k$, $(P^k, V^k)$ is $IP$ with perfect completeness and soundness error $2^{-\Omega (k)}$

Proof:
\begin{enumerate}
    \item $V^k$ is efficient  (composition of polynomials).
    \item Perfect Completeness - due to Perfect  Completeness of the original protocol - each iteration is guaranteed to succeed thus the protocol always does.
    \item Soundness: Let $x\in L, P^*$. We will show that $Pr[(P^*, V^k)(x)=1]\leq 2^{-k}$.
    Denote by $E_i$ the event that $V^k$ accepts in the $i$'th iteration. Thus:
    \[
        Pr[E_1\wedge E_2\wedge\dots E_k]=\prod_{i=1}^kPr[E_i|E_1\wedge E_2\wedge\dots E_k]
    \]
    Claim: $Pr[E_i|E_1\wedge E_2\wedge\dots E_k]\leq 0.5$.\\
    Proof: Assume toward a contradiction that $Pr[E_i|E_1\wedge E_2\wedge\dots E_k]>0.5$
    We design a prover $P^{**}$ that convinces $V$ with up to $>0.5$:\\
    $P^{**}$ emulates $(P^*, V)$ for iterations $1...i-1$ until the event $E_1\wedge E_2\wedge\dots E_k$ happens and then runs $(P^*,V)$ as the $i$'th iteration. Since the run of $(P^*,V)$ for the $i$'th iteration only happens under the condition $Pr[E_i|E_1\wedge E_2\wedge\dots E_k]$ - the probability for $(P^*,V)$ to happen on the $i$'th iteration is exactly $Pr[E_i|E_1\wedge E_2\wedge\dots E_k]>0.5$ but this is also the probability for $Pr[(P^{**},V)(x)=1]$. Contradiction.\\
    
    Thus:
    \[
        Pr[E_1\wedge E_2\wedge\dots E_k]=\prod_{i=1}^kPr[E_i|E_1\wedge E_2\wedge\dots E_k]\leq \prod_{i=1}^k0.5=2^{-k}
    \]
\end{enumerate}

\section{Graph Isomorphism \& $IP$ Example}
\subsection{Graph Isomorphism - Definition}
The graphs $G_1=(V,E_1), G_2=(V,E_2)$ are isomorphic or $(G_1,G_2)\in GI$ if $\exists\pi:V\longrightarrow V$ s.t. $(u,v)\in E_1 \iff (\pi(u), \pi(v))\in E_2$.\\
More simply - two graphs are isomorphic if they are identical up to a renaming of their vertices.\\

$GNI$ is the set of pairs of graphs which are isomorphic.\\
\begin{itemize}
    \item Claim (no proof): $NP\subseteq IP$.\\
    \item Claim (proof sketch): $GI\in IP$.\\
    By finding the permutation $\pi$ it is easy to check the $GI$ condition over a given $(G_1,G_2)$ thus we have an $NP$ relation, meaning $GI\in NP$.\\
    \item Claim (no proof): $GNI\in IP$.
\end{itemize}