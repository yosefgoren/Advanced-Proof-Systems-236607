\documentclass{article}
% basics
\usepackage{amsfonts}
\usepackage{amsmath}
\usepackage{enumitem}
\usepackage{float}
\usepackage{graphicx}
\usepackage{hyperref} 
\usepackage[labelfont=bf]{caption}
\usepackage[english]{babel}
\usepackage{amsthm}

\theoremstyle{definition}
\newtheorem{definition}{Definition}[section]

\newtheorem{theorem}{Theorem}
\newtheorem{lemma}[theorem]{Lemma}
\newtheorem{corollary}{Corollary}[theorem]

% unique math expressions:  
\usepackage{amsmath}
\DeclareMathOperator*{\andloop}{\wedge}
\DeclareMathOperator*{\pr}{Pr}
\DeclareMathOperator*{\approach}{\longrightarrow}
\DeclareMathOperator*{\eq}{=}

% grey paper
\usepackage{xcolor}
% \pagecolor[rgb]{0.11,0.11,0.11}
% \color{white}

% embedded code sections
\usepackage{listings}
\definecolor{codegreen}{rgb}{0,0.6,0}
\definecolor{codegray}{rgb}{0.5,0.5,0.5}
\definecolor{codepurple}{rgb}{0.58,0,0.82}
\lstdefinestyle{mystyle}{
    commentstyle=\color{codegreen},
    keywordstyle=\color{magenta},
    numberstyle=\tiny\color{codegray},
    stringstyle=\color{codepurple},
    basicstyle=\ttfamily\footnotesize,
    breakatwhitespace=false,         
    breaklines=true,                 
    captionpos=b,                    
    keepspaces=true,                 
    numbers=left,                    
    numbersep=5pt,                  
    showspaces=false,                
    showstringspaces=false,
    showtabs=false,                  
    tabsize=2
}

\newcommand{\fld}[0]{\mathbb{F}}
\newcommand{\onz}[0]{\{0,1\}}


\title{Advanced Proof Systems - Course Material}
\author{Yosef Israel Goren}
\begin{document}
\maketitle

\part*{Lecture 1}
\section{Recap}
\subsection{$P$ - Polynomial (Class)}
$L\in\{0,1\}^*$ is also in $P$ if there exists an efficient algorithm which decides it.

\subsection{$NP$ - Nondeterministic Polynomial}
$L\in\{0,1\}^*$ is also in $NP$ if there exists an efficient algorithm $V$ and a polynomial $p$ which follow:

\begin{enumerate}
    \item Completeness: $\forall x\in L, \exists y: V(x,y) = 1 \wedge |y|<p(|x|)$
    \item Soundness: $\forall x\notin L \forall y: V(x,y) \neq 1 \vee |y| \geq p(|x|)$
\end{enumerate}

\subsection{$PPT$ - Probobalistic Polynomial Time}
This is a class of algorithms which must run in time polynomial to the size of their input, but also - must be capable of randomization, or 'flipping coins'.

\subsection{$IP$ - Interactive Proof}
A key difference between an Interactive Proof and a proof for an NP proof is that the latter necessarily requires the prover to provide the verifier with something he can use to prove the truth of the calim to others.\\

We denote $(P,V)(x)$ to be the output of $V$ (verifier) after the interaction between $P$ and $V$ on the input $x$. Both $P$ and $V$ can be thought of as $PPT$ algorithms or programs which are capable of communicating with one another.\\

These interactions are often described with an interaction diagram:
\begin{itemize}
    \item $P$ sends to $V$ something
    \item $V$ sends to $P$ something else
    \item \dots
    \item $V$ accepts iff \dots
\end{itemize}

Formal Definition:
We say that $L\in IP$ if there exists a polynomial algorithm $V$, an unbounded algorithm $P$ and some constant $c\in (0.5, 1]$ s.t.
\begin{enumerate}
    \item Completeness: $\forall x\in L, Pr[(P,V)(x)=1]>c$
    \item Soundness: if $x\notin L,\forall P^*\in\mathbf{M}, Pr[(P^*,V)(x)=1]<1-c$
\end{enumerate}
Note: $\mathbf{M}$ denotes the set of turing machines.

%todo: add definition for Perfect Completness

\section{Equivalence of $IP$ separation constants}
\subsection{Iterative Runs}
Given an $IP$ protocol $(P,V)$, let $(P^k,V^k)$ be the protocol obtained by running $(P,V)$ k times sequentially. $V^k$ accepts iff in all iterations $V$ accepted.

\subsection{Lemma}
if $(P,V)$ is $IP$ with perfect completeness then for every polynomial $k$, $(P^k, V^k)$ is $IP$ with perfect completeness and soundness error $2^{-\Omega (k)}$

Proof:
\begin{enumerate}
    \item $V^k$ is efficient  (composition of polynomials).
    \item Perfect Completeness - due to Perfect  Completeness of the original protocol - each iteration is guaranteed to succeed thus the protocol always does.
    \item Soundness: Let $x\in L, P^*$. We will show that $Pr[(P^*, V^k)(x)=1]\leq 2^{-k}$.
    Denote by $E_i$ the event that $V^k$ accepts in the $i$'th iteration. Thus:
    \[
        Pr[E_1\wedge E_2\wedge\dots E_k]=\prod_{i=1}^kPr[E_i|E_1\wedge E_2\wedge\dots E_k]
    \]
    Claim: $Pr[E_i|E_1\wedge E_2\wedge\dots E_k]\leq 0.5$.\\
    Proof: Assume toward a contradiction that $Pr[E_i|E_1\wedge E_2\wedge\dots E_k]>0.5$
    We design a prover $P^{**}$ that convinces $V$ with up to $>0.5$:\\
    $P^{**}$ emulates $(P^*, V)$ for iterations $1...i-1$ until the event $E_1\wedge E_2\wedge\dots E_k$ happens and then runs $(P^*,V)$ as the $i$'th iteration. Since the run of $(P^*,V)$ for the $i$'th iteration only happens under the condition $Pr[E_i|E_1\wedge E_2\wedge\dots E_k]$ - the probability for $(P^*,V)$ to happen on the $i$'th iteration is exactly $Pr[E_i|E_1\wedge E_2\wedge\dots E_k]>0.5$ but this is also the probability for $Pr[(P^{**},V)(x)=1]$. Contradiction.\\
    
    Thus:
    \[
        Pr[E_1\wedge E_2\wedge\dots E_k]=\prod_{i=1}^kPr[E_i|E_1\wedge E_2\wedge\dots E_k]\leq \prod_{i=1}^k0.5=2^{-k}
    \]
\end{enumerate}

\section{Graph Isomorphism \& $IP$ Example}
\subsection{Graph Isomorphism - Definition}
The graphs $G_1=(V,E_1), G_2=(V,E_2)$ are isomorphic or $(G_1,G_2)\in GI$ if $\exists\pi:V\longrightarrow V$ s.t. $(u,v)\in E_1 \iff (\pi(u), \pi(v))\in E_2$.\\
More simply - two graphs are isomorphic if they are identical up to a renaming of their vertices.\\

$GNI$ is the set of pairs of graphs which are isomorphic.\\
\begin{itemize}
    \item Claim (no proof): $NP\subseteq IP$.\\
    \item Claim (proof sketch): $GI\in IP$.\\
    By finding the permutation $\pi$ it is easy to check the $GI$ condition over a given $(G_1,G_2)$ thus we have an $NP$ relation, meaning $GI\in NP$.\\
    \item Claim (no proof): $GNI\in IP$.
\end{itemize}
\part*{Tutorial 1}
\section{$IP$ and $NP$}
Claim:
$NP\subseteq IP$.\\
Proof: let $L\in NP$.
There exists some $NP$ relation $R$ for $L$, with an efficient algorithm $M_R$ which decides it.\\
Now define an $IP$ protocol:\\
\begin{itemize}
    \item Both $P$ and $V$ get $x$.
    \item If $x\in L$ P find $y$ s.t. $(x,y)\in R$, and send it to $V$. Otherwise send $\epsilon$.
    \item $V$ checks if $(x,y)\in R$ by running $M_R$ (known to be efficient) and accepts iff $M(x,y)$ accepts.
\end{itemize}
\begin{enumerate}
    \item Completeness: If $x\in L$, such $y$ must exist ($NP$ definition) thus $P$ will find it, and $V$ will have $(x,y)\in R$ so $M_R$ and $V$ will accept.
    \item Soundness: If $x\notin L$, there is no $y$ which such that $(x,y)\in R$ so no matter what any $P^*$ sends - $M_R(x,y)$ rejects and so $V$ rejects too.
 \end{enumerate}

\section{Similar Proof Systems}
\subsection{Arthur-Merlin - $AM$}
\begin{itemize}
    \item Both parties get some input $x$.
    \item Arthur sends Merlin some randomized $\alpha$.
    \item Merlin sends back some $\beta$.
    \item Arthur accepets according to some $PPT$ algorithm which is a function of $x,\alpha$ and $\beta$ (usually denoted $A(x,\alpha,\beta)$).
\end{itemize}

\subsection{Merlin-Arthur - $MA$}
\begin{itemize}
    \item Both parties get input $x$.
    \item Merlin sends $\beta$ to Arthur.
    \item Arthur generates some random value $\alpha$.
    \item Arthur accepts according to some $PPT$ algorithm which is a function of $x,\alpha,\beta$.
\end{itemize}

Theorem: $MA\subseteq AM$.\\
Proof: Let $L\in MA$.
WLOG (without loss of generality), $L$ has a $MA$ protocol with perfect completeness (we will come back to this assumption later in the course).\\
Denote by $p(n)$ the length $\beta$ ($|\beta|\leq p(n)$).\\

Using repetition we can get to any protocol with perfect completeness and a soundness error of $2^{-p(n)-1}$ (as seen in lecture).\\
Sketching the repeated protocol would look like:
\begin{itemize}
    \item $M'$ sends $\beta$ to $A'$
    \item $A'$ sends back a list of $\alpha$ values (as many as there are repetitions).
    \item $A'$ decides wether to accept.
\end{itemize}
This is because there is not reason for the prover's proof ($\beta$) to change due to different sampling of $\alpha$, so it is always the same and can be sent once. So the length of Merlin's message does not change in the repeating protocol.\\

Now consider the same $M',A'$ protocol but where $A'$ sends the aggregated $\alpha$ before $M'$ sends $\beta$.

Claim: This new protocol is $AM$.
Proof:
\begin{enumerate}
    \item Completeness: $\forall x\in L, M'$ sends the same $\beta$ without looking at $\alpha$:
    \[Pr[(M',A)(x)=1]=Pr[A(x,\alpha,\beta)=1]=1\]
    \item Soundness: Let $x\notin L$, fix $M^*$.
    Consider:
    \[
        Pr[(M^*,A')(x)=1]
        =Pr[\exists\beta\in\{0,1\}^p: A'(x,\alpha,\beta)=1]
    \]\[
        =\cup_{\beta\in\{0,1\}^p} Pr[A'(x,\alpha,\beta)=1]
        \leq_{UB} 2^p2^{-(p+1)}=\frac{1}{2}
    \]
    Note: $UB$ denotes Union Bound.
\end{enumerate}

\part*{Lecture 2}
\section{Recap}
\subsection{Remarks on $GI$}
\begin{itemize}
    \item For any graph $G$, the set of all $(G,G')$, where $G'$ is isomorphic to $G$ is an equivalence relation.
    \item For any graph $G$: $(\pi(G))_{\pi\leftarrow\mathbf{U}}\approx\mathbf{Iso(G)}$ - meaning uniformly sampling a permutation (over the set of nodes) of $G$ and applying it to $G$ will yield a uniformly sampled graph from the same equivalence partition.
\end{itemize}
\subsection{$GNI\subseteq IP$}
To prove this we will present an interactive protocol that runs on input $G_0, G_1$:
\begin{itemize}
    \item $V$ samples $b\leftarrow \{0,1\},\pi\leftarrow \{\texttt{permutations}\}$.
    \item $V$ sends $G=\pi(G_b)$.
    \item $P$ checks: if $G\approx G_0$, $b'=0$ else $b'=1$.
    \item $P$ sends $b'$ to $V$.
    \item $V$ accepts iff $b=b'$.
\end{itemize}
Now we prove that this interactive protocol is indeed an interactive proof for $GNI$:
\begin{enumerate}
    \item Soundness:
    Let $G_0\approx G_1$ and let $P^*$ be a (possibly) cheating prover.\\
    If $b=0$ then $G$ is uniformly distributed over all graphs isomorphic to $G_0$.\\ If $b=1$ then $G$ is uniformly distributed over all graphs isomorphic to $G_1$, but due to transitivity, this is the same as the prior case, meaning in any case, $G$ is distributed uniformly over the equivalence to $G_0$.\\
    Since this is the case, no information has been passed from $V$ to $P^*$ by sending $G$ ($P^*$ could have just sampled a graph from $Iso(G_0)$ itself!), $P^*$ could not do more than guess $b$, and it's probability for success is no more than $\frac{1}{2}$.
    \item Completeness: If the graphs are not isomorphic, $G\approx G_0$ iff $b=0$, which will mean $b=b'$ is always the case and $V$ always accepts.
    \item Runtime: $V$ does nothing of high complexity, $P$ is unbounded.
\end{enumerate}


\section{Equivalence between $IP$ and $AM[Poly]$}
\subsection{Lemma: Set Lower Bound Protocol}%TODO: add protocol and proof

\subsection{Theorem}
\begin{itemize}
    \item $IP[k]$: $k$-round interactive proof.
    \item $AM[k]$: $k$-round public coin interactive proof.
\end{itemize}
Claim (no proof):
$AM[k]\subseteq IP[k]$.\\
Theorem:
$IP[k]\subseteq AM[k+2]$.\\
Corollary: $IP=AM[Poly]$.\\
Theorem Proof:\\
Let $L$ be a language with an $IP$ which satisfies the following assumptions:
\begin{enumerate}
    \item $k=2$.
    \item Perfect Completeness.
    \item There is a set of $N$ of possible verifier messages which is constant for any specific protocol input $x$.
    \item The message $m$ sent by $V$ is uniformly distributed over the $N$ possible messages that could be sent.
    \item The soundness error is $\frac{1}{100}$.
\end{enumerate}
We want to show that $L$ is in $AM[k+2]$.\\
First, note how our $IP$ can be described as follows when running on input $x$:\\
\begin{itemize}
    \item $V$ samples some $r\leftarrow\{0,1\}$.
    \item $V$ calculates some $m=V_1(x,r)$ for some efficient function $V_1$.
    \item $V$ sends $m$ to $P$.
    \item $P$ sends back some $\pi$.
    \item $V$ accepts iff $V_2(x,\pi,r)=1$ for some efficient function $V_2$.
\end{itemize}
Define:
\[R_m=\{r'\mid m=V_1(x,r')\}, S_m=\{r'\in R_m\mid V_2(x,\pi,r')=1\}\]
Now consider an alternative protocol, which is the same as the one above, only $V$ also samples $r'\leftarrow R_m$ and accepts iff $V_2(x,\pi, r')$.

Note how $|R_m|=\frac{2^l}{N}$, and if $x\in L$ then $S_m=R_m$.
Claim (missing proof):\\%TODO add proof
if $x\in L$ then $\mathbf{E}(|S_m|)=\frac{2^l}{N}$ and otherwise $\mathbf{E}(|S_m|)\leq \frac{1}{100}\frac{2^l}{N}$.\\

\subsection{Pairwise Independent Hash Functions}
Definition:
A set of hash functions $\{h:X\leftarrow Y\}$ is pairwise independent if for all $x_1, x_2\in X$ where $x_1\neq x_2$ and $y_1,y_2\in Y$, they satisfy:
\[Pr_{h\leftarrow H}[h(x_1)=y_1\wedge h(x_2)=y_2]=\frac{1}{|Y|^2}\]
Construction:\\
Let $\mathbf{F}$ be a finite field.
Given $a,b\in\mathbf{F}: h_{a,b}(x)=ax+b$.\\
Let $H=\{h_{a,b}\mid a,b\in\mathbf{F}\}$.\\
Construction proof:\\
Let $x_1,x_2,y_1,y_2$ as described in definition.\\
\[
    Pr_{a,b}[ax_1+b=y_1\wedge ax_2+b=y_2]
    =Pr_{a,b}[
\begin{pmatrix}
x_1 & 1\\
x_2 & 1
\end{pmatrix}
\begin{pmatrix}
a\\
b
\end{pmatrix}
=
\begin{pmatrix}
y_1\\
y_2
\end{pmatrix}
]\]
\[
 =Pr_{a,b}[
\begin{pmatrix}
a\\
b
\end{pmatrix}
=
\begin{pmatrix}
x_1 & 1\\
x_2 & 1
\end{pmatrix}^{-1}
\begin{pmatrix}
y_1\\
y_2
\end{pmatrix}
]=\frac{1}{|\mathbf{F}|^2}\]
For the last transition: Note how the right value of the equality (in the probability) is just a constant value, so we just need to randomly sample $a,b$ to be some spesific values within a finite field.

\part*{Tutorial 2}
\section{Perfect Completeness?}
We have seen in the lecture how an $IP$ can be reduced to an $AM$ proof, or in other words; a public coin $IP$. Here, we would like to show how any $AM$ can be reduced to an $AM$ with perfect completeness (which is also an $IP$ with perfect completeness).\\
This means that $\forall L\in IP$, it must have a public coin, perfectly complete interactive proof.\\
To construct the reduction, we start with the following $z$-round public coin protocol ($AM[z]$), which runs on input $X$:
\begin{itemize}
    \item $A$ samples $\alpha\leftarrow \{0,1\}^{rc}$, and sends it to $M$.
    \item $M$ calculates $\beta=M(X,\alpha)$ and sends it to $m$.
    \item $A$ accepts iff $A(X,\alpha,\beta)=1$.
\end{itemize}
Where we assume completeness error $\epsilon>0$:
\[\forall x\in L:
Pr[(M,A)(X)=1]\geq 1-\epsilon\]
\[\Rightarrow \forall x\in L:
Pr[\exists \beta: A(X,\alpha,\beta)=1]\geq 1-\epsilon\]

%TODO: add intermidiate protocol (k=1)

Now, we want to use it to construct an equivalent with perfect completeness; for that end, consider the alternative protocol:
\begin{itemize}
    \item $M'$ samples $s_1,s_2,...,s_k\leftarrow \{0,1\}^{rc}$. s.t. (*)
    \[\forall \alpha \{0,1\}^{rc},\exists i\in [k]: s_i\oplus \alpha\notin REJ\].
    \item $M'$ sends $s_1,s_2,...,s_k$ to $A'$.
    \item $A'$ samples $\alpha \leftarrow \{0,1\}^{rc}$ and sends it to $M'$.
    \item $M'$ calculates $\forall i: \beta_i=M(X,s_i\oplus\alpha)$ and sends it.
    \item $A'$ accepts iff $\left(\exists i:A(X,s_i\oplus\alpha, \beta_i)\right)=1$
\end{itemize}

Lemma 1:
if $x\in L$ then pre-processing succeeds.\\
We denote $\bar{s}=(s_1,s_2,...,s_k)$. We say that $\bar{s}$ is 'good' if it satisfies (*).\\
To prove the lemma, we can show that:
\[
    \exists \bar{s}:\bar{s}\texttt{ is good }
\]
We will show this by first showing:
\[
    Pr[\bar{s}\texttt{ is good }]>0
\]
To start off:
\[
    Pr_{\bar{s}}[\bar{s}\texttt{ is not good }]
    =Pr_{\bar{s}}[\exists\alpha, \forall i:s_i\oplus\alpha\in REJ]
\]\[
    =Pr_{\bar{s}}[\bigcup_\alpha\left(\forall i: s_i\oplus\alpha\in REJ\right)]
%\]\[
%    =Pr_{\alpha_1, \alpha_2,...,\alpha_k\leftarrow \mathbf{U}}[\bigcup_\alpha\left(\forall i: \alpha_i\in REJ\right)]
\leq 2^{rc}\cdot \epsilon^k
\]
From here we can see that for $k$ of at-least $rc$, the probability is less then 1, meaning that the complementory probability is more than 0, meaning :
\[
    \exists \bar{s}:\bar{s}\texttt{ is good }
\]
Completeness is trivial given Lemma 1.\\
Soundness of the protocol can be found in notes of Tutorial 2 in the website.
\part*{Lecture 3}
\section*{Public Coin Protocol For Set Size}
\subsection*{Lemma (Set LB)}
Exists a protocol $(P,V)$ such that:
Given membership access to $s\subseteq U, t\subseteq \mathbb{N}$:
\begin{enumerate}
	\item Completeness:
	\[|s|\geq t\Rightarrow Pr[V \texttt{ accepts}]\geq \frac{2}{3}\]
	\item Soundness:
	\[\forall P^*, |s|\leq \frac{t}{100}\Rightarrow Pr[V \texttt{ accepts}]\leq \frac{1}{3}\]
\end{enumerate}

% TODO: add pairwise independent function definition

\subsection*{The Protocol}
\begin{enumerate}
	\item $V$ samples $h\leftarrow H$ where $H=\{h:U\rightarrow [t]\}$
	\item $P$ finds $x\in S$ s.t. $h(x)=17$ and sends it to $V$ (if none exist, $P$ fails).
	\item accepts iff $x\in S$ and $h(x) = 17$
\end{enumerate}

\subsection*{Proof}
For $x\in U, h\in H$, denote $E_x$ if $h(x)=17$.\\
Soundness: $|S|\leq\frac{t}{100}$.\\
\[
	Pr_{h\leftarrow H}[\exists x\in S, h(x)=17]
	=Pr_{h\leftarrow H}[\bigcup_{x\in S}E_x]
	\leq\sum_{x\in S}Pr[E_x]
	=\frac{|S}{t}
	\leq\frac{1}{100}
\]
Completeness: $|S|=t$.\\
\[
	Pr_{h\leftarrow H}[\exists x\in S, h(x)=17]
	=Pr_{h\leftarrow H}[\bigcup_{x\in S}E_x]
	\geq\sum_{x\in S}Pr[E_x]-Pr_{x,x'\in S, x<x'}[E_x\cap E_{x'}]
\]\[
	=\frac{|S|}{t}-\binom{|S|}{2}\frac{1}{t^2}
	\geq \frac{|s|}{t}-\frac{|s|^2}{2t^2}=\frac{1}{2}
\]

\section*{Zero Knowlage Proofs ($ZKP$)}
\subsection*{$ZKP$ protocol for $GI$}
Given input $G_0, G_1$:
\begin{itemize}
	\item $P$ finds a permutation $\psi\in S_n$ s.t. $G_0=\psi(G_1)$.
	\item $P$ uniformly samples a random permutation $\pi\leftarrow S_n$.
	\item $P$ sends $G=\pi(G_0)$ to $V$.
	\item $V$ samples $b\leftarrow \{0,1\}$ and sends it to $P$.
	\item $P$ defines $\sigma=\pi$ if $b=0$ else $\sigma=\pi\circ\psi$ and sends it to $V$.
	\item $V$ accepts iff $\sigma(G_b)=G$.
\end{itemize}

\subsection*{$HV-ZKP$ definition}
An interactive proof $(P,V)$ is a honest-verifier zero knolage proof of $L\subseteq\{0,1\}^*$ if
there exists a $PPT$ algorithm $S$ called 'the simulator' for which:
$\forall x\in L:$
The following distributions are 'similar':
\begin{itemize}
	\item $View^{x,p}(x):=(\texttt{the input }x,\texttt{ randomized coins }r, \texttt{ communications transcript})$
	\item $S(x) \texttt{ (the output of the simulator)}$
\end{itemize}
Here 'similar' can mean two different things:
\begin{itemize}
	\item If it means identiacal, meaning these two distributions are the same distribution,
	we call the protocol a \textbf{perfect $ZKP$}.
	\item If it means that the distributions are computationally indistinguishable, we call the protocol a \textbf{statistical $ZKP$},
	the definition for 'computationally indistinguishable' is in the next tutorial.
\end{itemize} 


The idea here is that since the verifier $V$ could have seen
everything which is seen during the interactive protocol $(P,V)$ by running the simulator $S$,
the verifier did not actually learn anything from it.

The reason this is 'honset-verifier' is because this definition
does not catch the case where the verifier 'cheats' and does not
run the protocol $V$, and might be able to extract information $P$ by doing it.


\subsection*{$ZKP$ definition}
An interactive proof $(P,V)$ is a zero knowlage proof for $L$
if:
\[\forall x,V^*,\exists S: View^{P,V^*}(x)\approx S(x)\]

Like to 'similar' from the honest verifier definition,
'$\approx$' can mean either 'identical' or 'computational indistinguishable';
with the former being a perfect $ZKP$ and the latter being a statistical $ZKP$. We sometimes denote computational indistinguishability by $X\approx_CY$
and perfect indistinguishability with $X\approx_PY$.\\

In words, the difference from the honest-verifier zero ZKP is
that now the simulator should be able to simulate anything that
ANY verifier protocol manages to extract, meaning that regardless
of the verifier protocol - no information is extracted.

\subsection*{$HZ-ZKP$ proof for $GI$ protocol above}
\[
	View:=(G_0,G_1,G,b,\sigma), G=\sigma(G_b)	
\]
Define the following protocol (simulator); on input $G_0, G_1$ do:
\begin{itemize}
	\item Sample $\sigma \leftarrow S_n$
	\item Sample $b\leftarrow \{0,1\}$
	\item Define $G=\sigma(G_b)$
	\item return $(G_0, G_1, G, b, \sigma)$
\end{itemize}
Now consider each cell in the output tuple of the two distributions ($View$ and $S(G_0,G_1)$),
$G_0,G_1$ are constant and always the same. $b\sigma$ are sampled uniformly from
their ranges in both cases.
So far there is no corralation between the cells.
$G$ is defined exactly given the rest of the variables meaning
that since the rest of the variables are the same for both distributions
$G$ is also the same. Furthermore, since the correlations were identical so far,
the new correlations are too.

These proofs often (such as in this case) follow the pattern of
first showing that given a subset of the cells of the distribution tuples
are the same due to being uniformly sampled respectively ($(G_0,G_1,b,\sigma)$ in this case),
then showing how the rest of the cells ($G$ in this case)
are a deterministic function of the other cells,
and so the addition of these new cells does not impare with the 
distributions being the same.\\
Formally, if we have distributions $D_1, D_2$ and a (deterministic) function $f$:
\[
	D_1\approx D_2
	\Rightarrow
	(D_1, f_{d\leftarrow D_1}(d))\approx (D_2, f_{d\leftarrow D_2}(d))
\]

\subsection*{$NP\subseteq ZKP$ (proof concept)}
In the lecture, proof by picture is shown for $3COL\in NPC$
has a zero knowlage proof. The main course of the proof consists of
the prover commiting to a spesific permutation of the colors on
a graph (without showing the verifier what these are), and allowing
the verifier to choose a sepsific edge and see that indeed
the coloring on both sides of that edge differ.
For each iteration a cheating prover has some chance of failing due
to the verifier haveing a chance to guess an edge where the two sides
have the same color. Furthermore, the protocol is zero knowlage
since the actual colors seen in each iteration are just two random colors.
\part*{Tutorial 3}
\section*{Computational \& Statistical Zero Knowledge}
\subsection*{Distinguisher and Advantage - Definition}
A distinguisher $D$ is a probabilistic polynomial time algorithm;
it receives an input $w$ and tries to decide if $w\in X$ or $w\in Y$.\\
The advantage of $D$ over $X, Y$ is defined as:
\[
	adv_D(X,Y):=|Pr_{w\leftarrow X}[D(w)=1]-Pr_{w\leftarrow Y}[D(w)=1]|	
\]

\subsection*{Negligable Functions}
A negligable function is a function
$\epsilon:\mathbb{N}\rightarrow\mathbb{R}$ s.t.
\[\forall c\in\mathbb{R}, \epsilon(n)=o(\frac{1}{n^c})\]
This is equivalent to saying that:
\[\forall\texttt{polynomial }p(n), \epsilon(n)\leq \frac{1}{p(n)}\]

\subsection*{Computational Indistinguishability}
Let $X,Y$ be two ensambles of distributions, meaning
that each of them consists of a seiries of distributions:
\[
	X=\{X_1,X_2,...\}, Y=\{Y_1,Y_2,...\}
\]
We say that $X$ is computationally indistinguishable from $Y$ if
for every distinguisher $D$ there exists a negligable function $\epsilon$
such that:
\[
	\forall n\in\mathbb{N}, adv_D(X_n,Y_n)\leq \epsilon(n)
\]

\section*{$ZKP$ for $3COL$ (more formally)}
Given an input graph $G=([n],E)$:
\begin{itemize}
	\item $P$ finds a $3$-coloring $\phi$ of $G$.
	\item $P$ samples a permutation $\xi$ over $[3]$ (the colors).
	\item $\forall v\in[n]$, 'put $\xi(v)$ in a box' $\beta_v$ and send it to $V$.
	\item $V$ samples $(u,v)=e\leftarrow E$ and sends it to $P$.
	\item $P$ sends the keys for $\beta_v$ and $\beta_u$ to $V$.
	\item $V$ accepts iff colors 'inside' $\beta_v, \beta_u$ are different.
\end{itemize}
To formalize the idea of these 'boxes' we define the 
notion of a commitment scheme:\\
A \textbf{commitment scheme} is a pair of PPT algorithms: $commit, check$ with the following
syntax:
\begin{itemize}
	\item $commit(b;r)\rightarrow c$
	\item $check(c,b,r)\rightarrow \{0,1\}$
\end{itemize}
And which satisfy the following conditions:
\begin{enumerate}
	\item \textbf{Computationally Hiding}: $commit(0)\approx_C commit(1)$.
	\item \textbf{Perfectly Binding}: There is no $n_0,r_0,r_1$ and $C^*$ s.t.
	\[
		check(C^*,0,r_0)=check(C^*,1,r_1)=1\texttt{ (1 means accept)}	
	\]
\end{enumerate}
\part*{Lecture 4}
\section*{$coNP\subseteq IP$}
\subsection*{Arithmetization}
We interested in a reduction from a $coNP$ problem to an arithmetic
problem.\\

\underline{The Sumcheck Problem:}\\
Parameters: a finite field $\mathbb{F}$, and $n,d\in\mathbb{N}$.\\
Input: $Q:\mathbb{F}^n\leftarrow\mathbb{F}$, $\alpha\in\mathbb{F}$.
Problem: does $\sum_{x\in\{0,1\}^n}Q(x)=\alpha$ ?\\

\underline{The reduction}
We will be reducing from $coNP$ to the sumcheck problem
by reducing $3-CNF$ to it (since $3-CNF$ is $coNP$-complete).\\
Let $\phi\in 3-CNF$ with $n$ variables and $m$ clauses.\\
We will start the construction by translating each building block
of $3-CNF$ formulas and expressing it in terms of polynomials:
\begin{enumerate}
	\item $\phi(x_1,x_2,...,x_n)=x_1\longrightarrow p(x_1,x_2,...,x_n)=x_1$
	\item $x_1\wedge x_2 \longrightarrow x_1\cdot x_2$
	\item $x_1\wedge \neg x_3\longrightarrow x_1\cdot(1-x_3)$
	\item $(x_1)\vee (x_2)=\neg((\neg x_1)\wedge(\neg x_2))\longrightarrow 1-(1-x_1)\cdot(1-x_2)$
\end{enumerate}
Lemma: for every $3-CNF$ formula $\phi$ on $m$ clauses and a finite field $\mathbb{F}$,
there exists a ploynomial $p:\mathbb{F}^n\rightarrow\mathbb{F}$ with degree $O(m)$ s.t.
$\phi=p$.\\
Furthermore, given $\phi$ and $z\in\mathbb{F}^n$, $p(z)$ can be
evaluated in $poly(n,m,log(|\mathbb{F}|))$ time.
\part*{Tutorial 4}
%here should be a detailed proof for 3COL in ZKP.
\part*{Lecture 5}
\section*{Reminder}
\underline{Lemma (sumcheck):}\\
There exists an IP between $P$ and $V$ where $P$
gets as input a polynomial $Q:\mathbb{F}^n\rightarrow\mathbb{F}$,
of individual degree $d$ and a value $\alpha\in\mathbb{F}$
and $V$ gets 'oracle access' to $Q$ and $\alpha$ explicitly.

\begin{itemize}
	\item Completeness:
	\[\left[
		\sum_{x\in\{0,1\}^n}Q(x)=\alpha
	\right]
	\Rightarrow V \texttt{ accepts w.p. }1\]
	\item Soundness:
	\[\left[
		\sum_{x\in\{0,1\}^n}Q(x)\neq\alpha
	\right]
	\Rightarrow\left[
		\forall P^*, Pr[V \texttt{ accepts with }P^*]\leq\frac{nd}{|\mathbb{F}|}
	\right]\]
\end{itemize}

\underline{Lemma:}
For every $3CNF; \phi:\onz^n\rightarrow\onz^n$
on $m$ clauses and field $\fld$ %TODO: complete the lemma from picture taken of board.

\section*{Probobalistically Checkable Proofs - $PCP$}
\subsection*{Definition}
We say that a $PPT$ machine $V$ is a $PCP$ verifier
for $L$ if:
\begin{enumerate}
	\item Completeness:
	\[
		\left[x\in L\right]
		\Rightarrow \exists\pi: Pr[V^\pi(x)=1]=1
	\]
	\item Soundness:
	\[
		\left[x\notin L\right]
		\Rightarrow \forall\pi^*: Pr[V^{\pi^*}(x)=1]\leq\frac{1}{2}
	\]
	\item Parameters:
	\begin{itemize}
		\item Query complexity - number of queries to $\pi$, denoted with $q$.
		\item Proof length - $|\pi|$, denoted with $l$.
		\item Randomness length - number of random bits sampled by the verifier, denoted with $r$.
	\end{itemize}
\end{enumerate}
We denote $PCP(q,r)$ the class of languages
with $PCP$ with query complexity $q$ and randomness length $r$.

\subsection*{Initial Properties \& Conclusions}
\begin{enumerate}
	\item
	Claim: $l\leq q\cdot 2^r$
	\item 
	Claim: $NP\subseteq PCP(poly, 0)$, more formally:
	\[
		NP\subseteq \bigcup_{p\in poly}PCP(p,0)	
	\]
	\item
	Claim: if $L$ has an interactive proof in which each prover/verifier message has length $a/b$ respectively.
	and with $k$ rounds, then $L$ has a $PCP$
	with length $a\cdot 2^{kb}$ and query complexity $a\cdot k$.
	\item
	Claim: $PCP(q,r)$ has $NP$ proof of length $q\cdot 2^r$.
\end{enumerate}
3. results with that any $L\in PSPACE$ has a proof
of exponential length that can be checked by a polynomial verifier, with a polynomial
number of queries.

\section*{PCP theorem: $NP = PCP(O(1), O(log(n)))$}
We will see the proof of this claim in a few steps:
\begin{itemize}
	\item $NP \subseteq PCP(O(1), poly(n))$
	\item $NP\subseteq PCP(O(log(n)^2), O(log(n)))$
	\item Combining the two claims above to get the theorem.
\end{itemize}

\subsection*{Hardness of Approxiamtion (Application)}
$GapSAT_\epsilon$: Accept satisfiable $CNF$ formulas,
and reject formulas s.t. every assignment satisfies at most $\epsilon$ clauses.
(here $\epsilon \in[0,1]$ is the proportion of clauses satisfied).\\

\underline{Theorem:}\\
\[
	\exists\epsilon >0: GapSAT_\epsilon\in NPC
\]
\underline{Proof:}\\
Let $L\in NP$.\\
Given $x\in\onz^n$, if $x\in L$,
there exists a $PCP$ proof $\pi$ for $x$ (The PCP theorem).\\

For a given $x, p$ and $V_{PCP}$, define $V_{x,p}:\onz^q\rightarrow\onz$
as follows:\\
There exists a CNF computing $V_{x,p}$ with $2^q$ clauses.\\
Consider the following CNF:
$
	\bigwedge_{p}V_{x,p}
$
\begin{itemize}
	\item if $x\in L$, it is easy to see that $V_{x,p}$ is satisfied.
	\item 
	if $x\notin L$ for any assignment $\pi$ for half of $p$'s $V_{x,p}(\pi)=0$
	so overall, at-least $\frac{1}{2}\frac{1}{2^q}$ of the clauses 
	are not satisfied.
\end{itemize}
With $\epsilon=1-\frac{1}{2^{q+1}}$, we have that
$L\in GapSAT_\epsilon$.
\part*{Tutorial 5}
\section*{Error-Correcting Codes (ECC)}
\subsection*{Motivation}
The purpose of error correcting codes is
to handle systems where information is passed through
a noisy channel, some information needs to be 
written and read from a medium, but there 
there is no guarenteed that the information
will be read exactly as written - rather, the
are some weaker guarentees about the amount of
errors or changes that can occur between the information
that is written and the information that is read.

\subsection*{Introduction}
A simple solution to enable correcting errors would
be to repeat the written message multiple times - for example,
write the input 3 times, and if there is a conflict on
some bit - choose the majority.\\
It is easy to see how any single bit error 
can be corrected by this method.\\

\subsection*{Parameters for ECC}
For an error correction code $C:\fld^k\rightarrow\fld^n$
\begin{enumerate}
	\item Rate: $\frac{n}{k}$ (redundency $n-k$)
	\item Minimal Distance:
	\[
		min_{m_1\neq m_2}\Delta\left(C(m_1),C(m_2)\right)
	\]\[
		hw(x)=|\{i|x_i\neq 0\}|, 
		\Delta(x,y) = hw(x-y)	
	\]
\end{enumerate}

\subsection*{Linear Codes}
If $C:\fld^k\rightarrow\fld^n$ is a linear function, meaning:
\[
	\forall m_1,m_2\in\fld^k, \forall \alpha,\beta:	
	C(\alpha\cdot m_1+\beta\cdot m_2)=\alpha\cdot C(m_1)+\beta\cdot C(m_2)
\]
Then we get the following conclusions:
\begin{itemize}
	\item $C(0)=0$.\\
	Easy to prove by adding 0 to the argument of $C$.
	\item $d=min_{m\in\fld^k\setminus\{0\}}C(m)$.\\
	Proof:
	\begin{enumerate}
		\item 
		\[
			d=min_{m_1\neq m_2}\Delta (C(m_1), C(m_2))
			=min_{m_1\neq m_2}hw(C(m_1)-C(m_2))
		\]\[
			=min_{m_1\neq m_2}hw(C(m_1-m_2))
			\leq min_{m\neq 0}hw(C(m))
		\]
		\[
			d\leq min_{m\neq 0} \Delta(C(m), C(0))
			=min_{m\neq 0}hw(C(m)) 	
		\]
	\end{enumerate}
	\item Singlton Bound: $d\leq n-k+1$.\\
\end{itemize}

\section*{ECC important examples}
\subsection*{Hadamard Code}
Let $Had:\onz^k\rightarrow \onz^{2^k}$ where:
\[
	\forall m\in \onz^k: Had(m)_i=<m,i>	
\]
Then get get:
\begin{enumerate}
	\item Rate: $\frac{n}{k}=\frac{2^k}{k}$
	\item Absolute distance: $\frac{2^k}{2}$.\\
	Proof: $\frac{d}{n}$ of $Had$ is $\frac{1}{2}$.\\
	Claim: if $m\in\onz^k\setminus\{\overline{0}\}$ then
	\[
		Pr_{x\in\onz^k}[<m,x>=0]
		=Pr_{x\in\onz^k}[<m,x>=1]
		=\frac{1}{2}	
	\]
	Let $m\in\onz^k\setminus\{\overline{0}\}$,
	by claim from HW1, exactly half of the bits of $Had(m)$
	are $1$, and thus - $hw(Had(m))=\frac{1}{2}\cdot 2^k$
\end{enumerate}

\subsection*{Read-Solomon Codes}
Presume we map each input
to a polynomial: 
$m\longrightarrow \fld^k[x]$.\\
And let $m_1\neq m_2\longrightarrow p_1\neq p_2$.\\
Thus $p_1, p_2$ agree on at most $k-1$ points in $\fld^k$.\\

\[
	RS(m)=p(\alpha_1), p(\alpha_2), \dots, p(\alpha_n)	
	\Rightarrow d=n-k+1
\]


\part*{Lecture 6}
\section*{$NP\subseteq PCP(O(1),Poly(n))$ - cont.}
Hadamard Code: $Had:\{0,1\}^n\rightarrow \{0,1\}^n$, Relative distance $\frac{1}{2}$.\\

\subsection*{Hadamard $PCP$}
Quadratic Equations: $GF(2)$.\\
$n$-variables, $m$-equations:
\[
	x_1x_2+x_7+x_{19}x_1=1
\]\[
	x_2+x_{38}x_{19}=0	
\]
Each such equation can be described by $a\in\onz^{n^2}$ for the
coeeficiants of $x_i$'s. So the whole set of equations
can be described with a matrix $A\in\onz^{n^2\times m}$ for
all equations and $b\in\onz^m$ for the constant terms.\\

Now given such $A$ and $b$, consider the problem of deciding:
\[
	\exists u\in\onz^n A(u\otimes u)=b	
\]
Where here $\otimes$ denotes the open product:
\[
	u\otimes u=\begin{matrix}
		u_1u_1&u_1u_2&\dots&u_1u_n\\
		u_2u_1&u_2u_2&\dots&u_2u_n\\
		\vdots&\vdots&\ddots&\vdots\\
		u_nu_1&u_nu_2&\dots&u_nu_n
	\end{matrix}
\]

\subsection*{Verifier sketch}
Now we can define a Hadamard $PCP$ verifier for this problem:\\
The $PCP$ proof string: given $u\in\onz^n$ s.t. $A(u\otimes u)=b$,
a proof is $(Had(u)\in\onz^{2^n}, Had(u\otimes u)\in\onz^{2^{n^2}})$.\\

Given $A,b$ we want to check the following:
\begin{enumerate}
	\item The proof string consists of some $(Had(u),Had(U))$.
	\item Check $U=u\otimes u$.
	\item $A\cdot U = b$.
\end{enumerate}

\subsection*{Checking 2}:\\
To start off, we will show how 2 can be verified:\\
For that end, consider the following indentity:\\
$\forall a,b,c,d\in\fld^n$:
\[
	<a\otimes b, c\otimes d>
	=\sum_{i,j}a_ib_jc_id_j
	=\sum_ia_ic_i\sum_jb_jd_j
	=<a,c>\cdot<b,d>
\]
Now consider it with $u, r, s$:\\
\[
	<u\otimes u, r\otimes s>
	=<u,r>\cdot<u,s>
\]

Moreover, for the purposes of our verifier,
we know that if 2 is satisfied, the following
identity should be satisfied (Soundness):
\[
	<U, r\otimes s>
	=<u,r>\cdot<u,s>
\]
And if $\forall r,s\in\onz^n$ we have the identity - then
2 must be satisfied (Completness).\\

So to check $2$, our verifier will choose $r,s\in\onz^n$
and test \[
	<U, r\otimes s>
	=<u,r>\cdot<u,s>
\]

Formally;
\begin{itemize}
	\item Soundness: let $U\neq u\otimes u$.
	\[
		\pr_{r,s}[<U,r\otimes s>=<u,r>\cdot<u,s>]	
		=\pr_{r,s}[<U,r\otimes s>=<u\otimes u, r\otimes s>]
	\]\[
		=\pr_{r,s}[<u-u\otimes u, r\otimes s>=0]\leq_* \frac{3}{4}
	\]
	To understand *, we can see that for any $A\neq 0$,
	the probability that $r\in\onz^n$ is $0$ is negligable,
	thus $r^TA\neq 0$ and so $r^tAs$ is a multiplication
	between a non-zero matrix and a uniformly sampled vector - which
	is also uniformly distributed. Hence:
	\[
		\pr_{r,s}[<A,r\otimes s>=0]
		=\pr_{r,s}[r^tAs=0]
		=1-\pr_{r,s}[r^tAs\neq 0]
	\]\[
		=1-\pr_{r,s}[r^tA\neq 0]\cdot \pr_{r,s}[(r^tA)s\neq 0\mid r^tA\neq 0]
		=1-\frac{1}{2}\cdot\frac{1}{2}
	\]
\end{itemize}

\subsection*{Checking 3}:\\
Let $r\in\onz^m$ Consider:
\[
	<r^TA,u\otimes u>=r^Tb
	\Leftrightarrow
	r^TA\cdot(u\otimes u)=r^Tb
\]
if $A\cdot u\otimes u=b$ then 1 is satisfied (perfect completness).\\
if $A(u\otimes u)\neq b$ then (soundness):
\[
	\pr_r[r^TA\cdot(u\otimes u)=r^Tb]
	=\pr_r[r^T(A\cdot u\otimes u-b)=0]=\frac{1}{2}=\frac{3}{4}	
\] 

\subsection*{Checking 1}:\\
To check 1, we start off by showing a theorem.\\
\underline{Theorem (Linearity Testing):}\\
There exists a probobalistic algorithm $A$ that
takes $O(1)$ quaries to a function $f:\onz^n\rightarrow \onz$ and:
\begin{enumerate}[label=(\alph*)]
	\item if $f$ is linear then $A$ accpets w.p. $1$.
	\item if $\Delta(f,L_{lin-n})\geq 0.01$ then $A$ accepts w.p. $\leq \frac{1}{2}$.
\end{enumerate}

\underline{Proof:}\\
Definition 1: $f:\onz^n\rightarrow \onz$ is
linear if $\exists c\in\onz^n$ s.t. $f(x)=<c,x>$.\\
Definition 2: $f:\onz^n\rightarrow \onz$ is
linear if $\forall x,y: f(x+y)=f(x)+f(y)$.\\
Now we show that the two definitions are equivalent:\\
\begin{itemize}
	\item $1\Rightarrow 2:$
	Let $f:\onz^n\rightarrow \onz$ for which  $\exists c\in\onz^n$ s.t. $f(x)=<c,x>$.\\
	\[
		f(x+y)=<c,x+y>=<c,x>+<c,y>=f(x)+f(y)	
	\]
	\item $2\Rightarrow 1:$
	Let $f:\onz^n\rightarrow \onz$ for which $\forall x,y: f(x+y)=f(x)+f(y)$.\\
	\[
		f(x)=f(\sum_{i=1}^n x_i\bar{e}_i)=\sum_{i=1}^n f(x_i\bar{e}_i)
		=\sum_{i=1}^n f(x_ie_i)
		=\sum_{i=1}^nx_if(e_i)%TODO: complete this...
	\]
\end{itemize}

\underline{Lemma:}\\
\[
	\Delta(f,L_{lin-n})>\delta
	\Rightarrow \pr_{x,y}[f(x+y)\neq f(x)+f(y)]>1-\delta
\]

\underline{Fourier Analysis:}\\
\[
	\onz\rightarrow \{1,-1\}, b\rightarrow (-1)^b
\]\[
	f:\onz^n\rightarrow \onz, g:\{1,-1\}^n\rightarrow \{1,-1\}
\]
$f$ is linear iff $g(x)=\prod_{i\in s}x_i$, 
(because mapping is homomorphism).

\subsection*{Forurier Analysis Crash Course:}
Consider the set of functions: $g:\{1,-1\}^n\Rightarrow\mathbb{R}$.\\
This set is a vector space over $\mathbb{R}$.\\
A basis for this vector space is the \emph{Fourier Basis} $\{g_1,g_2,...,g_n\}$.
$g_i(x)$ can be calculated as follows:\\
let $b_0,b_1,...,b_{n-1}$ be the binary representation of $i$.\\
\[
	g_i(x)=\prod_{j=0}^{n-1}x_j^{b_j}
\]

\part*{Tutorial 6}
\section*{Error Correcting Codes - cont.}
\subsection*{Locally Testable Codes}
\underline{Goal:}
Given $\pi\in\onz^n$, we want to decide if
$\exists x: C(x)=\pi$.\\
Moreover, we are interested in doing so
while only reading a fraction of $x$'s bits.\\
Doing so exactly, is impossible due to 
the ability of an adveresery to change a single random
bit from a correct value.\\

\underline{Relaxed Requirment:}
\begin{enumerate}
	\item if $\exists x: C(x)=\pi$ accept.
	\item reject if $\pi$ is $\delta$-far from $C$.
\end{enumerate}


Distance from code:
\[
	\Delta(\pi,C)=\min_x\Delta(\pi,C(x))
	\Rightarrow \pi \text{ is } \delta-\text{far from } C: \Delta(\pi,C)\geq \delta	
\]\[
	\Rightarrow \pi \text{ is } \delta-\text{close from } C: \Delta(\pi,C)\leq \delta
\]

\underline{Definition:}
$C$ is $q$-local $\delta$-testable if:\\
$\exists A\in PPT$ s.t.
given oracle access to $\pi$:
\begin{enumerate}
	\item if $\exists x: C(x)=\pi$ then $ A^\pi=1$.
	\item if $\Delta(\pi, C)>\delta$ then:
	\[
		\pr[A^\pi=1]<\frac{1}{2}
	\]
	and $A$ makes (up to) $q$ queries to $\pi$.
\end{enumerate}

\subsection*{Locally Decodable Codes}
A code that allows us to decode a bit from original message by
making a small amount of queries even if there's a small
fraction of corrupted bits.

\underline{Definition:}\\
a code is $(q,\delta,\epsilon)$-locally decodable if:\\
$\exists A\in PPT$ that given $i\in[n]$ and oracle access to $\pi$ s.t.\\
\[
	\left[\exists x: \Delta(\pi, C(x))\leq \delta\right]
	\Rightarrow \pr[A^\pi(i)=x_i]\geq 1-\epsilon	
\]
and $A$ makes $q$ queries to $\pi$.

\underline{Lemma:}\\
$Had$ is $()$-locally decodable.

\underline{Proof:}\\
Show some alg $A^\pi(i)$:
\begin{enumerate}
	\item $\beta\leftarrow \onz^k, \gamma =\beta+e_i$\\
	(where for ex. $e_3=00100...0$ ).
	\item $\pi_\beta+\pi_\gamma$
\end{enumerate}

\underline{Correctness:}\\
let $x: \Delta(C(x),\pi)\leq\delta$.
\[
	\pr[A^\pi(i)=x_i]
	=\pr_\beta[\pi_\beta+\pi_{\beta+e_i}]
	\geq\pr[\pi_\beta=Had_\beta(x)\wedge \pi_{\beta+e_i}=Had_{\beta+e_i}(x)]
\]\[
	\pr[\pi_\beta\neq Had(x)\vee \pi_{\beta+e_i}\neq Had_{\beta+e_i}(x)]
\]\[
	\leq \pr[\pi_\beta\neq Had_\beta(x)]+\pr[\pi_\gamma\neq Had_\beta(x)]
	\leq 2\delta
\]

\part*{Lecture 10}


\subsection{Doubly Efficient Interactive Proofs}
\begin{definition}[Logspace-Uniform]
	We say that \{$C_n\}_{n\in\mathbb{N}}$ is logspace uniform if exists
	a turing machine that on input $1^n$ outputs $C_n$.
\end{definition}

\begin{theorem}
	If $L$ is accepted by logspace uniform circuits of depth $d(n)$ and of size $poly(n)$,
	then $L$ has an Interactive proof which the prover runs in $poly(n)$
	time and the verifier runs in $(n+d)polylog(n)$ time with $d\cdot polylog(n)$ rounds.
\end{theorem}

\end{document}