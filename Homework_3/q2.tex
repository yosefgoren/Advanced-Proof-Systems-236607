\subsection{A Characterization}
\underline{$C_1\otimes C_2\subseteq\mathbb{M}$}:\\
Let $\mathbb{M}\subseteq\fld^{n_1\times n_2}$ be the set of matrices 
such that all rows are codewords of $C_2$ and all columns are codewords of $C_1$.\\

As defined, we know that $C_i:\fld^{k_i}\longrightarrow\fld^{n_i}$. Furthermore,
we know $C_i$ is a linear function. This means that $C_i$ can be represented as a
matrix, or in other words, we can describe $C_i(v)=C_i\cdot v, \forall v\fld^{k_1}$.\\
Furthermore, if we have a matrix $A\in\fld^{n_1\times X}$,
applying $C_2$ to all of it's columns is equivalent to multiplying $A$ by $C_2$ as in $C_2\cdot A$.
If we want to do the equivalent operation for the rows, we can transpose $A$, then transpose the result
after multiplication, meaning that for any matrix $A\in\fld^{x\times n_2}$,
applying $(C_1\cdot(A^T))^T$ is equivalent to applying $C_1$ to all of $A$'s rows.\\

Finally, if we want to first apply $C_2$ to all columns, then apply $C_1$ to all rows of
a matrix $A\in\fld^{n_1\times n_2}$ - we can describe this operation as: $C_1(C_2A^T)^T$, meaning (*):
\[
    (C_1\otimes C_2)(A)=C_1(C_2A^T)^T=C_1(AC_2^T)=(C_1A)C_2^T=(C_2(C_1A)^T)^T
\]

Let $B\in C_1\otimes C_2$. By def. $\exists A: (C_1\otimes C_2)(A)=B$.
From (*), this means $B=C_1(AC_2^T)$, or in other words $\exists D=AC_2^T$,
s.t. $B = C_1D$ and so each column of $B$ is the result of the
$C_1$ on a column of $D$ - meaning it is a leagal codeword of $C_1$.\\
Similarly, from (*) we know $\exists D=C_1A$ s.t. $B=(C_2D^T)^T$ - meaning
every row is a legal codeword of $C_2$.\\
Hence $B\in\mathbb{M}$.
In other words; $C_1\otimes C_2\subseteq\mathbb{M}$.

\underline{$\mathbb{M}\subseteq C_1\otimes C_2$}:\\

\underline{$C_1\otimes C_2=\mathbb{M}$}:\\
An anti-symmetric relation $R\subseteq A\times B$ is
a relation that obays:
\[
    \forall a\in A, \forall b\in B: (a,b)\in R \wedge (b,a)\in R \Rightarrow a=b
\]
Lemma: '$\subseteq$' is anti-symmetric:\\
Let $A,B$ be two sets s.t. $A\subseteq B\wedge B\subseteq$.
By def. this means:
\[\forall a\in A, a\in B\wedge \forall b\in B, b\in A\]
Thus $\forall x: x\in A\iff x\in B$ and so $A=B$.\\
Hence, $\subseteq$ is anti-symmetric.\\

Thus, since we have shown that $C_1\otimes C_2\subseteq\mathbb{M}$
and also $\mathbb{M}\subseteq C_1\otimes C_2$, we can conclude that
$C_1\otimes C_2=\mathbb{M}$.\\


\textbf{But most importantly, we have shown we remember Matka!}

\subsection{Rate}
We know the input size of $C_1\otimes C_2$ is $k_1\cdot k_2$ and the outpus size
is $n_1\cdot n_2$. Thus the rate is:
$r=\frac{k_1\cdot k_2}{n_1\cdot n_2}=\frac{k_1}{n_1}\cdot\frac{k_2}{n_2}=r_1\cdot r_2$

\subsection{Distance}
Let $A\in\mathbb{M}$.\\
By def. $hw(A)\geq \delta_1n_2n_1$, $hw(A)\geq \delta_2n_1n_2$.
Thus:
\[
    hw(A)\geq max(\delta_1n_2n_1, \delta_2n_2n_1)
    \Rightarrow hw(A)\geq max(\delta_1, \delta_2)\cdot n_1n_2
    \Rightarrow \frac{hw(A)}{n_1n_2}\geq max(\delta_1, \delta_2)\geq \delta_1\delta_2
\]
Since $\delta =\frac{hw(A)}{n_1n_2}$ we have $\delta \geq \delta_1\delta_2$.
\subsection{Tensor of Reed-Solomon}
\subsection{Sumcheck for Tensor Codes}
% Let $|S| := h$ and $\{s_1,...,s_k\} := S$.\\
Given an input $c,\alpha$:\\
Define $w\in\fld^n$ s.t. for all $\lambda\in S$,
\[w[\lambda]:=\sum_{i_2,i_3,...,i_d\in S}c(\lambda, i_2,...,i_d)\]
Let $v:=C_{base}^{-1}(w)$.\\

\textbf{The protocol:}
\begin{itemize}
    \item \underline{Verifier}:
    \begin{enumerate}
        \item Recive $v$ as $v'$.
        \item Compute $w':=C_{base}(v')$.
        \item Verify $\sum_{\lambda\in S}w'[\lambda]=\alpha$.
        \item If $d=1$:
        \begin{enumerate}
            \item Verify $w'=c$ (by reading $c$).
        \end{enumerate}
        \item Otherwise:
        \begin{enumerate}
            \item Sample $\lambda\leftarrow^\$[n]$.
            \item Verify $w'[\lambda]=w[\lambda]$, recursively
                with $\alpha$ being $w'[\lambda]$,\\
                $c$ being $c(\lambda,:,...,:)$
                (the left $c$ is the parameter of the recursively
                invoked protocol and the right $c$ is the paraneter of the invoking protocol)
                and $d$ being $d-1$.
        \end{enumerate}
    \end{enumerate}
    \item \underline{Prover}:
    \begin{enumerate}
        \item Send $v$.
        \item Recursively follow the verifier's protocol.
    \end{enumerate}
\end{itemize}

\textbf{Correctness:}
\begin{itemize}
    \item \underline{Complexity}:\\
        The base cases's complexity is $O(log|\fld|\cdot n)$.\\
        Each recusive step also requires $O(n)$ opeartions,
        where each one is an operation in the field $\fld$ and thus can be done with
        complexity $log|\fld|$, so the total complexity of a step is $O(n\cdot log|\fld|)$
        and the total complexity is $O(n\cdot d\cdot log|\fld|)$.\\
    \item \underline{Completness}:\\
        The base case is immidiate. And recursively:\\
        If $c, \alpha$ satisfy the condition and $P$ sends $v'=v$.\\
        So $w'=w$.
        As we know $w$ satisfies the formula and so does $w'$ and the verifier passes step 2.,
        with the inductive assumption for completness - the verifier will also
        accept for $w'[\lambda]$ for any $\lambda\in[n]$ and the verifier passes 4.b.\\
        
    \item \underline{Soundness}:\\
        In the case $d=1$ the checking is exhaustive, so the protocol is perfectly sound,
        so the soundness error is $1\geq 1-\delta$.\\
        
        Let us assume a soundness error $1-\delta^{d-1}$ for the protocol
        of order $d-1$.\\
        Now consider an instance of protocol with order $d$ on input $c,\alpha$ that do
        not satisfy the condition.\\

        If $v'=v$, the verifier will disqualify the proof w.p. 1 at step 2 (*).\\

        If $v'\neq v$:
        \begin{enumerate}
            \item Since $v'\neq v$, $w'\neq w$.
            \item Since $\delta$ is the minimal relative distance of $C_{base}$:
                \[
                    \forall a,b\in C_{base}: a\neq b,
                    \pr_{\lambda\leftarrow [n]}[a(\lambda)\neq b(\lambda)]\geq\delta
                \]
            \item Due to 1. and 2.:
                \[
                    \pr_{\lambda\leftarrow [n]}[w'(\lambda)\neq w(\lambda)]\geq\delta
                \]
            \item If $\lambda$ is sampled s.t. it satisfies the expression at 3., the verifier will thus disqualify the proof w.p.
            at-least $1-\delta^{d-1}$ at step 4.b. due to the inductive assumption.
            \item Thus the chance that the verifier will accept is no more than $\delta\cdot \delta^{d-1}=\delta^d$ (**).
        \end{enumerate}

        Now consider the total probability of the verifier accepting the proof:
        \[
            \pr_{\lambda\leftarrow[n]}[(V,P)(c,\alpha)=1]
        \]\[
            = x\cdot\pr_{\lambda\leftarrow[n]}[(V,P)(c,\alpha)=1\mid v'=v]
                +(1-x)\cdot\pr_{\lambda\leftarrow[n]}[(V,P)(c,\alpha)=1\mid v'\neq v]
        \]\[
            \leq \pr_{\lambda\leftarrow[n]}[(V,P)(c,\alpha)=1\mid v'=v]
                +\pr_{\lambda\leftarrow[n]}[(V,P)(c,\alpha)=1\mid v'\neq v]
        \]\[
            =_{(*)} 0 +\pr_{\lambda\leftarrow[n]}[(V,P)(c,\alpha)=1\mid v'\neq v]
        \]\[
            \leq_{(**)} 1-\delta^d
        \]
        Hence we have a soundness error of $1-\delta^d$.
\end{itemize}