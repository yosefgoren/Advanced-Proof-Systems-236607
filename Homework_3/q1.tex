\subsection{}
Denote the concatenation of $m_1$ and $m_2$
with $m_1||m_2$.\\
Let $L\subseteq\onz^*$ be a language with a $k-prover$,
$2-message$ $MIP$.\\
Since this protocol is a $2-message$ $MIP$,
we know the structure of communications: in the
first round the verifier sends a message to all provers,
and in the second round each prover sends a message to the
verifier.

Let $x\in L$, in the run of the protocol on $x$,
Denote the message sent by the $V$
to $P_i$ with $m_{q,i}$,
denote the message sent
by $P_i$ to $V$ with $m_{r,i}$.

The set of possible values of $m_{q,i}$ is
bounded with it's maximal size. Each such value
will yield an appropriate $m_{r,i}$ response,
which is independet of any other $m_{q,j}$ 'queries'
sent by the verifier.\\
Thus for each $i\in[n]$ we can 
define a function from query values to response values;
for any query value we can define the response value
to be $0$.
Thus we are left with a function $Resp_i:\onz^{l_V}\rightarrow \onz^{l_P}$,
where $\forall m_{q,i}\forall, Resp_i(m_{q,i})=m_{r,i}$.\\

For the purposes of accessing the correct response later we can also
pad the response values with $0$'s so that all of them are exactly $l_P$ in length.\\

Define the $PCP$ proof of $x$ as:
\[
	PCP_x:=Resp_0(0)||Resp_0(1)||\dots||Resp_0(2^{l_V}-1)
\]\[
	||Resp_1(0)||Resp_1(1)||\dots||Resp_1(2^{l_V}-1)
\]\[
	...
\]\[
	||Resp_{k-1}(0)||Resp_{k-1}(1)||\dots||Resp_{k-1}(2^{l_V}-1)
\]

Note $|Resp_i(j)|= l_P$, thus each line
in the definition of $PCP_x$ is equal to $l_p\cdot 2^{l_V}$.
And the whole size is equal to $k\cdot l_p\cdot 2^{l_V}$.\\

The $PCP$ verifier $V_p$ with be based on the $MIP$ verifier $V_m$.\\
On the run of $V_p(x,PCP_x)$, $V_p$ will first
use $V_m$ to ask it what queries to make.
For each query $m_{q,i}$ to $P_i$ made by $V_m$,
$V_p$ will look at:
\[
	PCP_x[l_P\cdot(2^{l_V}\cdot i+m_{q,i}):l_P\cdot(2^{l_V}\cdot i+m_{q,i})+l_P]
\]
Or in other words, the bits corresponsing to the response for
$m_{q,i}$ in $P_i$'s section of $PCP_x$.\\

After the $PCP$ verifier gets all these bits,
it gives them back to $V_m$ as the responses to the queries -
and accepts iff $V_m$ accepts.\\

\begin{itemize}
	\item Completness:\\
	If $x\in L$ - our $PCP_x$ is well defined.
	And for whatever query $V_m$ makes - it recives the exact
	response it should get from $P_i$ - thus since it will accept
	on $P_i$'s responses - it will accepts on the messages sent by $V_p$,
	meaning $V_p$ will accept on $x$.
	\item Soundness:\\
	Let there be a set of $PCP_x$ values (defined also on $x\notin L$).
	WLOG $\forall x,|PCP_x|=l_P$ - since it is easy for $V$ to check if that is the case.\\
	Thus we can esaly use this set of $PCP_x$'s to
	construct a set of 'mallicious' provers $P_0,...,P_{k-1}$:
	Each $P_i$ on query $m_{q,i}$ will response with:
	\[
		PCP_x[l_P\cdot(2^{l_V}\cdot j+m_{q,i}):l_P\cdot(2^{l_V}\cdot j+m_{q,i})+l_P]
	\]
	Now for each instance where $V_p(x,PCP_x)=1$, we have
	$(V_m,P_0,...,P_{k-1})(x)=1$ since the only way for $V_p$
	to accept is if $V_m$ does - and it runs on the same inputs (including randomizations) in both cases.\\
	Thus:
	\[
		\pr[V_p(x,PCP_x)=1]\leq \pr[(V_m,P_0,...,P_{k-1})(x)=1]\leq\frac{1}{2}
	\] 
\end{itemize}


\subsection{}
Let $L\subseteq\onz^*$ be a language with a $PCP$ verifier $V_p$ set of proofs
bounded by in length $m$.\\
Define the following $q-prover$ $MIP$ for it:\\
\textbf{The protocol on input $x$}:
\begin{itemize}
	\item \underline{Verifier $V_m$}:
		\begin{enumerate}
			\item sample $B\leftarrow^\$\onz$.
			\item if $B=0$ (verify $PCP$):
			\begin{enumerate}
				\item Get the set of queries $Q=\{Q_i\mid i\in[q]\}$ from $V_p$ on input $x$.
				\item For all $i\in[q]$, send $Q_i$ to $P_i$.
				\item Denote the bit returned by $P_i$ with $b_i$.
				\item Verify $V_p$'s acceptence on query results $\{b_i\mid i\in[q]\}$.
			\end{enumerate}
			\item if $B=1$ (Verify consistency):
			\begin{enumerate}
				\item Sample $r\leftarrow^\$[m]$.
				\item For all $i\in[q]$, send $r$ to $P_i$.
				\item Denote the bit returned by $P_i$ with $b_i$.
				\item Verify $b_i=b_j, \forall i,j\in[q]$.
			\end{enumerate}
		\end{enumerate}
	\item \underline{Prover $P_i$}:
		\begin{enumerate}	
			\item Recive an index $i$ from the verifier.
			\item Return $PCP_x[i]$.
		\end{enumerate}	
\end{itemize}



\textbf{Correctness}:
\begin{itemize}
	\item \underline{Complexity}:\\
		The integer representation of each query is of size $log(m)$,
		thus it is the length of the messages sent by $V_m$.
	\item \underline{Completness}:\\
		Let $x\in L$. Denote with $PCP_x[Q]$ the set of bits
		corresponding to the queries $Q$ in $PCP_x$.\\
		In the standard usage of the $PCP$ verifier $V_p$ - 
		it will recive $PCP_x[Q]$ as the responses to the queries $Q$ - 
		and since it has perfect completness (WLOG as seen previously in the course),
		it will accept.\\
		When $V_m$ invokes $V_p$ - it sends it the same $PCP_x[Q]$,
		thus $V_p$ accepts here too - and so does $V_m$.
	\item \underline{Soundness}:\\
		Let $x\notin L$.\\
		Let $\{P_i^*\mid i\in[q]\}$ be a set of (possibly mallicious) provers.\\
		
		Now we use these provers to construct $PCP_x$:\\
		$PCP_x$ has a section corresponding to each prover $P_i^*$, and in each section - 
		the $j$'th bit corresponds to the response of $P_i^*$ on query $j$ (as we
		have seen in the course, we can assume WLOG that the provers are determenitic).
		
		% Now we use these provers to construct $PCP_x$:\\
		% $PCP_x$ has a section corresponding to each prover $P_i^*$, and in each section - 
		% the $j$'th bit corresponds to the response of $P_i^*$ on query $j$ which
		% is the most probable. If there is more than one most probable response
		% select the minimal one.\\

		Claim:
		\[
			\pr[(V_m,P_1^*,...,P_q^*)(x)=1]
			\leq \pr[V_p(x, PCP_x)=1]
		\]
		Proof:
		As a shorthand denote $P_1^*,...,P_q^*$ as $PS^*$.\\

		% Denote with $C_{i,j}$ the event that $P_i$ returns the most
		% probable response on query $j$ and $c_{i,j}=\pr[C_{i,j}]$.
		% Since there are only two possible
		% responses (1 and 0) - $c_{i,j}\geq\frac{1}{2}$.\\

		In the event that $B=0$:\\
		Let $C:= \forall_{i,j\in [q]}b_i=b_j$ - meaning the event that
		all provers were consistent with one another. Let $c=\pr[C]$.\\

		All bits returned by $PS^*$ are the same ones $V_p$ would
		get by quering $PCP_x$ thus:
		\[
			\pr[(V_m,PS^*)(x)=1\mid B=0]	
			=\pr[V_p(x, PCP_x)=1]
		\]
		
		\[
			=\pr[(V_m,PS^*)(x)=1\mid B=0\wedge C]\pr[C]
			+\pr[(V_m,PS^*)(x)=1\mid B=0\wedge \neg C]\pr[\neg C]
		\]
		\[
			=c\pr[(V_m,PS^*)(x)=1\mid B=0\wedge C]
			+(1-c)\pr[(V_m,PS^*)(x)=1\mid B=0\wedge \neg C]
		\]
		\[
			=c\pr[V_p(x, PCP_x)=1]
			+(1-c)\pr[(V_m,PS^*)(x)=1\mid B=0\wedge \neg C]
		\]
		\[
			\leq c\pr[V_p(x, PCP_x)=1]+(1-c)\cdot 1
			=1+c(\pr[V_p(x, PCP_x)=1]-1)
		\]
		% In the event that $B=0$:\\
		% Let $C:= \bigwedge_{i\in [q]}C_{i,Q_i}$ - meaning the event that
		% all provers were consistent with $PCP_x$. Let $c=\pr[C]$.\\
		% Since we know the provers cannot communicate - the probabilities for their
		% selections are independent. Hence $c=\prod_{i\in [q]}c_{i,Q_i}$.
		% \[
		% 	\pr[(V_m,PS^*)(x)=1\mid B=0]	
		% \]
		% \[
		% 	=\pr[(V_m,PS^*)(x)=1\mid B=0\wedge C]\pr[C]
		% 	+\pr[(V_m,PS^*)(x)=1\mid B=0\wedge \neg C]\pr[\neg C]
		% \]
		% \[
		% 	=c\pr[(V_m,PS^*)(x)=1\mid B=0\wedge C]
		% 	+(1-c)\pr[(V_m,PS^*)(x)=1\mid B=0\wedge \neg C]
		% \]
		% \[
		% 	=c\pr[V_p(x, PCP_x)=1]
		% 	+(1-c)\pr[(V_m,PS^*)(x)=1\mid B=0\wedge \neg C]
		% \]
		% \[
		% 	\leq c\pr[V_p(x, PCP_x)=1]+(1-c)\cdot 1
		% 	=1+c(\pr[V_p(x, PCP_x)=1]-1)
		% \]

		In the event that $B=1$:\\
		Let $C':=\bigwedge_{i\in[q]}C_{i,r}$ and $c':=\pr[C']$.\\
		Thus $c'=\frac{1}{q}\sum_{r\in[q]}\prod_{i\in[q]}c_{i,r}$.\\
		\[
			\pr[(V_m,PS^*)(x)=1\mid B=1]
		\]
		\[
			\pr[(V_m,PS^*)(x)=1\mid B=1\wedge C']\pr[C']
			+\pr[(V_m,PS^*)(x)=1\mid B=1\wedge \neg C']\pr[\neg C']
		\]
		\[
			=c'\pr[(V_m,PS^*)(x)=1\mid B=1\wedge C']
			+(1-c')\pr[(V_m,PS^*)(x)=1\mid B=1\wedge \neg C']
		\]
		\[
			=c'\cdot 1+(1-c')\cdot 0=c'
		\]

		Thus:
		\[
			\pr[(V_m,PS^*)(x)=1]
			=\frac{1}{2}\pr[(V_m,PS^*)(x)=1\mid B=0]+\frac{1}{2}\pr[(V_m,PS^*)(x)=1\mid B=1]
		\]
\end{itemize}