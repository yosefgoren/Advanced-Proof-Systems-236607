\subsection{}
Denote the concatenation of $m_1$ and $m_2$
with $m_1||m_2$.\\
Let $L\subseteq\onz^*$ be a language with a $k-prover$,
$2-message$ $MIP$.\\
Since this protocol is a $2-message$ $MIP$,
we know the structure of communications: in the
first round the verifier sends a message to all provers,
and in the second round each prover sends a message to the
verifier.

Let $x\in L$, in the run of the protocol on $x$,
Denote the message sent by the $V$
to $P_i$ with $m_{q,i}$,
denote the message sent
by $P_i$ to $V$ with $m_{r,i}$.

The set of possible values of $m_{q,i}$ is
bounded with it's maximal size. Each such value
will yield an appropriate $m_{r,i}$ response,
which is independet of any other $m_{q,j}$ 'queries'
sent by the verifier.\\
Thus for each $i\in[n]$ we can 
define a function from query values to response values;
for any query value we can define the response value
to be $0$.
Thus we are left with a function $Resp_i:\onz^{l_V}\rightarrow \onz^{l_P}$,
where $\forall m_{q,i}\forall, Resp_i(m_{q,i})=m_{r,i}$.\\

For the purposes of accessing the correct response later we can also
pad the response values with $0$'s so that all of them are exactly $l_P$ in length.\\

Define the $PCP$ proof of $x$ as:
\[
	PCP_x:=Resp_0(0)||Resp_0(1)||\dots||Resp_0(2^{l_V}-1)
\]\[
	||Resp_1(0)||Resp_1(1)||\dots||Resp_1(2^{l_V}-1)
\]\[
	...
\]\[
	||Resp_{k-1}(0)||Resp_{k-1}(1)||\dots||Resp_{k-1}(2^{l_V}-1)
\]

Note $|Resp_i(j)|= l_P$, thus each line
in the definition of $PCP_x$ is equal to $l_p\cdot 2^{l_V}$.
And the whole size is equal to $k\cdot l_p\cdot 2^{l_V}$.\\

The $PCP$ verifier $V_p$ with be based on the $MIP$ verifier $V_m$.\\
On the run of $V_p(x,PCP_x)$, $V_p$ will first
use $V_m$ to ask it what queries to make.
For each query $m_{q,i}$ to $P_i$ made by $V_m$,
$V_p$ will look at:
\[
	PCP_x[l_P\cdot(2^{l_V}\cdot i+m_{q,i}):l_P\cdot(2^{l_V}\cdot i+m_{q,i})+l_P]
\]
Or in other words, the bits corresponsing to the response for
$m_{q,i}$ in $P_i$'s section of $PCP_x$.\\

After the $PCP$ verifier gets all these bits,
it gives them back to $V_m$ as the responses to the queries -
and accepts iff $V_m$ accepts.\\

\begin{itemize}
	\item Completness:\\
	If $x\in L$ - our $PCP_x$ is well defined.
	And for whatever query $V_m$ makes - it recives the exact
	response it should get from $P_i$ - thus since it will accept
	on $P_i$'s responses - it will accepts on the messages sent by $V_p$,
	meaning $V_p$ will accept on $x$.
	\item Soundness:\\
	Let there be a set of $PCP_x$ values (defined also on $x\notin L$).
	WLOG $\forall x,|PCP_x|=l_P$ - since it is easy for $V$ to check if that is the case.\\
	Thus we can esaly use this set of $PCP_x$'s to
	construct a set of 'mallicious' provers $P_0,...,P_{k-1}$:
	Each $P_i$ on query $m_{q,i}$ will response with:
	\[
		PCP_x[l_P\cdot(2^{l_V}\cdot j+m_{q,i}):l_P\cdot(2^{l_V}\cdot j+m_{q,i})+l_P]
	\]
	Now for each instance where $V_p(x,PCP_x)=1$, we have
	$(V_m,P_0,...,P_{k-1})(x)=1$ since the only way for $V_p$
	to accept is if $V_m$ does - and it runs on the same inputs (including randomizations) in both cases.\\
	Thus:
	\[
		\pr[V_p(x,PCP_x)=1]\leq \pr[(V_m,P_0,...,P_{k-1})(x)=1]\leq\frac{1}{2}
	\] 
\end{itemize}