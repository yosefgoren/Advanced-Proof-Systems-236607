Let $L\in NP$ be decided by a family of logspace uniform circuits
of size $\tilde{O}(n)$ - $\{C\}_{i\in\mathbb{N}}$.\\
Let $L\:=\{(x_1,...,x_k)\mid \forall i\in[k], x_i\in L\ wedge |x_i|=|x_1|=n\}$.\\
We also know that $t,k=poly(n)$.\\

Define the following family of circuits $C':=\{C'_i\}_{i\in\mathbb{N}}$:\\
The circuit $C'_i$ is defined as the concatenation
of $C_i$; $k$ times side by side, followed by
a tree of '$\wedge$' operators that combine
the outputs of all $C_i$ instances to a single output.\\

Since for each area of the input of $C_i'$
that input is processed by the circuit $C_i$,
and since the final result of $C_i'$ is to accept
iff $\forall j\in[k], x_j\in L$ - then $C'$
accepts $L'$ (note the sizes must match
also to be a valid input to $C_i'$).\\

Now consider the size and depth of $C'$;
it's size is $k$ times the size
of $C$'s circuits,
with the addition of the tree of '$\wedge$' operators.
Meaning the size is $\leq\tilde{O}(t)\cdot k+2\cdot log(k)$.
The depth of $C$ is bounded by it's size, hence
the depth of $C'$ is $\leq \tilde{O}(t)+log(k)$.\\

Now consider the theorem on compiling logspace uniform
circuits into IP's seen in lecture:\\
\begin{theorem}[logspace uniform circuits compiler]
	If $L$ is accepted by logspace uniform circuits of depth $d(n)$ and of size $poly(n)$,
	then $L$ has an Interactive proof which the prover runs in $poly(n)$
	time and the verifier runs in $(n+d)polylog(n)$ time with $d\cdot polylog(n)$ rounds.
\end{theorem}

Thus exists an IP proof for $L'$ based on $C'$,
s.t. the prover runs in time $poly(|x'|)$ where $x'\in L'$
or $poly(n\cdot k)$, there
are up to $d\cdot polylog(n\cdot k) rounds$ and the verifier
runs in time $(n\cdot k+d)polylog(n\cdot k)$.\\
Now all that is left is to evaluate the complexities:
\begin{itemize}
	\item \underline{Prover:}\\
		As required $poly(n\cdot k)\leq poly(n,k,t)$.
	\item \underline{Verifier:}\\
	\[
		(n\cdot k+d)polylog(n\cdot k)
		\leq (n\cdot k+\tilde{O}(t)+log(k))polylog(n\cdot k)
	\]\[
		= (n\cdot k+t\cdot polylog(t)+log(k))polylog(n\cdot k)
		= (n\cdot k+t\cdot polylog(poly(n))+log(k))polylog(n\cdot k)
	\]\[
		= (n\cdot k)polylog(n\cdot k)+t\cdot polylog(n)polylog(n\cdot k)+log(k)polylog(n\cdot k)
	\]\[
		= \tilde{O}(n\cdot k)+t\cdot polylog(n\cdot k)+log(k)polylog(n\cdot k)
	\]\[	
		=\footnote{
		$
			polylog(n\cdot k)=polylog(n\cdot poly(n))=
			poly(log(n\cdot n^c))=poly((c+1)log(n))=polylog(n)
		$	
		}
		\tilde{O}(n\cdot k)+t\cdot polylog(n)+log(k)polylog(n)
	\]\[
		=\tilde{O}(n\cdot k)+(t+log(k))polylog(n)
	\]
	\item \underline{Communications:}\\
	\[
		d\cdot polylog(n\cdot k)
		= \tilde{O}(t)polylog(k)polylog(n\cdot k)
		= \tilde{O}(t)polylog(n)\footnote{
			is the same as $(log(k)+t)polylog(n)$.
		}
		\]
\end{itemize}
