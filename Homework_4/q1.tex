\subsection{}
Given a graph with $n$ vertices, 
consider the following series of matrices:\\
\[
	M_k[i,j]:=
	\begin{cases}
		1 & \text{ exists a path from } i \text{ to } j \text{ with length } \leq k\\
		0 & \text{otherwise}
	\end{cases}
\]
Corollary: $M_1$ is the adjacency matrix of the graph.\\
Between any two matrices of size $n\times n$, define the following operation:
\[(A\otimes B)[i,j]:=\bigvee_{x\in [n]}\left(A[i, x]\wedge B[x, j]\right)\].\\

\underline{Lemma}:
\[\forall k\in[n]: M_k\otimes M_k=M_{2k}\].
\underline{Proof}:
Assume (by induction) for $M_k$
\footnote{formally this would not prove for
$k$ values that are not a power of $2$,
but we actually do not make use of those sizes anyways
since $\forall k\geq n, M_k=M_n$ and we only care about $M_n$.}.
Now, we need to prove for $M_{2k}$. Consider a path $l_{i,j}$
from $i$ to $j$ with length $\leq 2k$.\\
It can be decomposed into two paths of length $\leq k$;
The first path from $i$ to $x$ ($l_{i,x}$) and the second path from $x$ to $j$ ($l_{x,j}$).\\
Hence $M_{2k}[i,j]\Rightarrow \exists x\in[n]: M_k[i,x]\wedge M_k[x,j]\Rightarrow (M_k\otimes M_k)[i,j]$.\\
Conversely, if $(M_k\otimes M_k)[i,j]=1$, then there exists $x\in[n]$ such that $M_k[i,x]=1$ and $M_k[x,j]=1$,
which means that there exists a path from $i$ to $x$ of length $\leq k$ and a path from $x$ to $j$ of length $\leq k$.
So $(M_k\otimes M_k)[i,j]\Rightarrow M_{2k}[i,j]$.\\
Finally we have that $M_{k}\otimes M_{k}=M_{2k}$.\\
\newpage 
\underline{Algorithm}:
The algorithm which the circuit will follow is as follows:\\
\begin{algorithmic}
	\State $k\leftarrow 1$
	\State $M\leftarrow E$
	\While{$k<n$}
		\State $M\leftarrow M\otimes M$
		\State $k\leftarrow 2k$
	\EndWhile
	\State \Return $M[s,t]$.
\end{algorithmic}

Due to the lemma, at the end of each iteration - $M=M_k$,
this means that when the loop ends, $M=M_n$.
Thus $M[s,t]$ is the correct output.\\

\underline{Circuit}:
Here we assume the circuit recives $E,s,t$ as input, we assume $E$ is represented
in the adjacency matrix form.\\
The $\otimes$ operation can be implemented with depth $O(log(n))$
using a binary tree of the big $\vee$ gate.
Additionally $log(n)$ steps are required for $k:=n$ - each step
is a set of layers added to the circuit.\\
With $log(n)$ complexity we can implement a $MUX$ gate
to select $s,t$ from $M$ after the layer.\\
Hence the total depth of the circuit is $O(log(n)^2)$.
The maximum width of the circuit is $O(n^2)$.

\subsection{}
Let $L\in PSPACE$.\\
Hence exists a polynomial space nondeterministic turing machine $M_L$ that decides it.\\
Consider the set of configurations possible for $M_L$.
Since it's set of states is constant and the set of values for
it's tape is exponentionally bounded by a polynome - 
the set of possible configurations for it is bounded
exponentionally by a polynome.\\

Each such configuration can be represented as a vertex
in the graph of all possible configurations.
Moreover, each possible transition from one configuration to another
can be represented as an edge in the graph.
Denote this configuration graph as $G$.\\

Define the reduction $f$ as $f(x)=(G, s, t)$
where $s$ is the vertex corresponding to the initial
configuration of $M_L$ on the input $x$, and $t$
is vertex corresponding to the accepting configuration
\footnote{
	WLOG assume there is one accepting configuration,
	otherwise we can either construct a new turing
	machine that clears it's tape and then accepts,
	or add to the graph a new node ($t$) that
	is connected to all accepting configuration vertecies.
}.

\subsection{}
Let $L\in PSPACE$.\\
Due to the reduction from the previous question,
we know $L$ can be solved by a
poly turing machine for the ST-connectivity
problem as a graph with size exponential w.r. to $n$ (where $x\in L, n=|x|$).\\
Also, due to the first question, we have
a family of circuits that accept
or reject these exponential graphs 
with $size=poly(|G|)=poly(exp(n))=exp(n), depth=polylog(|G|)=polylog(exp(n))=poly(n)$.

\subsection{}
Let $L\in PSPACE$.\\
From the last question we have 
that it has an exponential size poly depth
circuit family \footnote{Exponential size'd family sounds amazing},
hence by the reduction seen in class -
it has an IP with verifier running 
in time $(poly(n)+n)polylog(exp(n))=poly(n)$.
Hence - $L\in IP$
\[
	\square
\]

\subsection{}
The complexity of the prover seen in class
is $poly(S)$, where $S$ is the size of the
circuit so the prover
time is $poly(exp(n))$ or just $exp(n)$.