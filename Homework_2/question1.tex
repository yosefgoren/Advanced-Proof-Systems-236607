\subsection{Polynomial Identity Lemma}
\underline{\textbf{Proof of lemma:}}\\
The multivariate polynomial degree as stated in the question (*):
\[
    Deg\left(\sum_{i=1}^m\prod_{i=j}^nx_j^{d_{i,j}}\right) = \max\biggl\{\sum_{j=1}^nd_{i,j}\mid i\in[n]\biggl\}
\]

Proof by induction.\\
Induction basis is immidiate from the fundemental theorem of algebra.\\

Assume for every nonzero $P=\sum_{i=1}^m\prod_{i=j}^nx_j^{d_{i,j}}$:
\[
    \pr_{x_1,...,x_{n-1}\in\mathbb{F}}\left[
        P(x_1,...,x_{n-1})=0
    \right]
    \leq \frac{Deg(P)}{|\mathbb{F}|}
\]

Let $P\in\mathbb{F}[x_1, x_2, ..., x_n]$ be a nonzero polynomial.\\
Denote $d=Deg(P)$.\\
By treating all $\{x_i\}_{i=1}^{}n-1$ as constants, $P$ can be described as a univariate polynomial in $x_n$:
\[P(x_1, x_2, ..., x_n)=\sum_{i=0}^dx_n^i\cdot P_i(x_1,...,x_{n-1})\]
Since $P$ is nonzero:
\[
    \exists i: P_i(x_1,...,x_{n-1})\neq 0
\]
Let: $k=\max\{i\mid P_i \neq 0\}$.\\
From (*): 
\[
    Deg(x_n^k\cdot P_k)=k+Deg(P_k)\leq Deg(P)=d
\]\[
    \Rightarrow Deg(P_k)\leq d-k
\]
Let $\{y_i\}_{i=1}^{n-1}\leftarrow^\$ \mathbb{F}^{n-1}$ and $x_n\leftarrow \mathbb{F}$ (sampled uniformly).\\
Define the occurances:
\begin{itemize}
    \item $SubRoot$: $P_k(y_1,...,y_{n-1})=0$
    \item $Root$: $P(y_1,...,y_{n-1},x_n)=0$
\end{itemize}
We are interested in showing that: $\pr[Root]\leq\frac{d}{|\mathbb{F}|}$.\\

From the induction assumption:
\[
    \pr[SubRoot]=\pr\left[P_k(y_1,...,y_{n-1})=0\right]
    \leq \frac{Deg(P_k)}{|\mathbb{F}|}
    \leq \frac{d-k}{|\mathbb{F}|}
\]
Define $P'(x_n)=P(y_1,...,y_{n-1},x_n)$, note that $P'(x_n)=0$ is equivalent to $Root$.\\
Assuming $\overline{SubRoot}$, $P'(x_n)$ is a univariate polynomial in $x_n$ of degree $k$.\\
Thus from the fundemental theorem of algebra:
\[
    \pr_{x_n\leftarrow \mathbb{F}}\left[P'(x_n)=0\right]
    \leq \frac{k}{|\mathbb{F}|}
\]
    In other words:
\[
    \pr[Root\mid \overline{SubRoot}]
    =\pr\left[P'(x_n)=0
    \mid \overline{SubRoot}\right]
    \leq \frac{k}{|\mathbb{F}|}    
\]
Now denote the case $Root$ to be the case where $P(y_1,...,y_{n-1}, x_n)=0$.\\
Denote $Root, SubRoot$ as $R,S$ for ease of notation, and we finally have:
\[
    \pr[R]=\pr[R\mid S]\cdot\pr[S]+\pr[R\mid \bar{S}]\cdot\pr[\bar{S}]
    \leq \pr[S]+\pr[R\mid \bar{S}]\leq \frac{d-k}{|\mathbb{F}|}+\frac{k}{|\mathbb{F}|}= \frac{d}{|\mathbb{F}|}
\]\[
    \Rightarrow \pr[R]\leq \frac{d}{|\mathbb{F}|}
    \Rightarrow \pr[P(y_1,...,y_{n-1},x_n)=0]\leq \frac{d}{|\mathbb{F}|}
\]

\underline{\textbf{Tightness:}}\\
Let $n\in\mathbb{N}, d\leq|\mathbb{F}|-1$.\\
We are interested in showing a polynomial for which the probability
of begin zero is exactly $\frac{d}{|\mathbb{F}|}$.\\
Denote $F:=|\mathbb{F}|$, $\{a_1,a_2,...,a_F\}:=\mathbb{F}$.\\
Define:
\[
    U(x):=\prod_{i=1}^d(x-a_i)    
\]
\[
    P(x_1,...,x_n):=U(x_n)    
\]
For each $a_i$, $U(a_i)=0\cdot\prod(\dots)$ thus each $a_i$ is a root of $U$,
additionally, if $U$'s argument $x$ is not $a_i$, then $U(x)$ is a product of nonzeros and thus $U(x)$ is nonzero.\\
So:
\[
    \pr_{x_1,...,x_n}[P(x_1,...,x_n)=0]=\pr_{x_n\leftarrow \mathbb{F}}[U(x_n)=0]
    =\pr_x[U(x)=0]=\frac{d}{F}
\]

\subsection{All-Zero Check with small Field}
Let $\fld$ be a finite field.\\
Define:
\[
    I_1(x,x')=1-x-x'+x\cdot x'+x\cdot x'
\]  
For any $n\in\mathbb{N}\setminus\{0,1\}$, define:
\[
    I_n(x,x')=
    \prod_{i\in[n]}I_1(x_i,x_i')
\]
For any polynomial $Q:\fld^n\rightarrow\fld$ and $z\in\fld$, Define:
\[
    Q_z(x):=Q(x)\cdot I_n(x,z), S_{Q,z}(x)=\sum_{x\in\onz^n}Q_z(x)
\]
Let $Q$ be the zero polynomial and $x,z\in\fld$,
then (*):
\[
    Q_z(x)=Q(x)\cdot I_n(x,z)=0
    \Rightarrow S_{Q,z}(x)=0
\] 
Also, note how $S_{Q,z}$ is a polynomial of individual degree $1$, and total degree $n$,
therefore, for any nonzero $Q$ and $z\in\fld$,
we can use the polynomial identity lemma to get (**):
\[
    \pr_{x\leftarrow\fld^n}\left[S_{Q,z}(x)=0\right]\leq\frac{n}{|\fld|}
\]


\underline{\textbf{Interactive Protocol:}}\\
Define our interactive protocol to be the sumcheck
protocol seen in class, with $\alpha = 0$
and applied on the $Q_z$ polynomial where $z$ 
is uniformly sampled from $\fld$.\\

\underline{\textbf{Completeness:}}\\
Let $Q$ be the zero polynomial and consider a specific run of $(P,V)(Q)$.\\
Let $z$ be the value sampled by $V$.\\
Due to (*), $S_{Q,z}(x)=0$ for all $x\in\fld^n$,
thus, due to the completness of the sumcheck protocol -
$V$ will accept.\\

\underline{\textbf{Soundness:}}\\
Let $P^*$ be a (possibly) mallicious prover, $Q$ be a nonzero polynomial and consider a specific run of $(P^*,V)(Q)$.\\
Denote $z$ to be the value sampled by $V$.\\

Consider the soundness error of the sumcheck protocol
given the arguments provided to it here; since $Q_z$ has
an individual degree of at most $d+1$, it will be $\frac{(d+1)n}{|\fld|}$ (***).\\

Thus:
\[
    \pr\left[(P^*,V)(Q)=1\right]
\]
\[
    =\pr\left[(P^*,V)(Q)=1\mid S_{Q,z}(x)=0\right]\cdot\pr\left[S_{Q,z}(x)=0\right]
\]
\[
    +\pr\left[(P^*,V)(Q)=1\mid S_{Q,z}(x)\neq 0\right]\cdot\pr\left[S_{Q,z}(x)\neq 0\right]
\]
\[
    =_{V def.}1\cdot\pr\left[S_{Q,z}(x)=0\right]
    +\pr\left[(P^*,V)(Q)=1\mid S_{Q,z}(x)\neq 0\right]\cdot\pr\left[S_{Q,z}(x)\neq 0\right]
\]
\[
    \leq_{(**),(***)}1\cdot\frac{n}{|\fld|}
    +\frac{n(d+1)}{|\fld|}\cdot \pr\left[S_{Q,z}(x)\neq 0\right]
\]
\[
    \leq_{\pr[\cdot]\leq 1}\frac{n}{|\fld|}
    +\frac{n(d+1)}{|\fld|}=O\left(\frac{n\cdot d}{|\fld|}\right)
\]

\underline{\textbf{Complexity:}}\\
As we have seen in class, running the sumcheck verifier is with complexity
$poly(n,d,log(|\fld|))$, and sampling a random element $z$
is also within those bounds. Hence the total runtime of our verifer 
is $poly(n,d,log(|\fld|))$.\\

Additionally, since our verifier does not make additional queries to $Q$ other 
then the ones done by the sumcheck verifier - only one query to $Q$ is made.

